% ============================================================================
%  Formatting module: здесь собраны все правила визуального оформления текста.
%  Структура файла повторяет основные сущности ГОСТ/университетских требований,
%  чтобы можно было быстро найти нужный блок и настроить его.
% ============================================================================

% --- Макет страницы и колонтитулы --------------------------------------------
% Управляет стилями страниц, включая выравнивание шапок и подвалов.
\makeatletter
\let\ps@plain\ps@fancy % Подчиняем первые страницы каждой главы общим правилам
\makeatother
\pagestyle{fancy}
\fancyhf{}
\fancyfoot[C]{\thepage}
\renewcommand{\headrulewidth}{0pt}
\renewcommand{\footrulewidth}{0pt}
\renewcommand{\baselinestretch}{1.5}
\newcommand{\headertext}[1]{\fancyhead[R]{\tiny{#1}}}

% --- Заголовки ---------------------------------------------------------------
% Заголовки отдельных уровней оформлены так, чтобы соответствовать нормам ПЗ:
% главы — центрированы и жирные, остальные уровни — выровнены по левому краю.
\titleformat{\chapter}[block]{\centering\normalfont\Large\bfseries}{\thechapter.}{1ex}{}{}
\titlespacing{\chapter}{0pt}{0em}{2em}

\titleformat{\section}[block]{\normalfont\large\bfseries}{\thesection}{1ex}{}{}
\titlespacing{\section}{0pt}{0em}{1ex}

\titleformat{\subsection}[block]{\normalfont\normalsize\bfseries}{\thesubsection}{1ex}{}{}
\titlespacing{\section}{0pt}{0em}{1ex}

% paragraph и subparagraph — в тексте, без отступов
\titleformat{\paragraph}[runin]{\normalfont\normalsize\bfseries}{\theparagraph}{0pt}{}{}
\titlespacing{\paragraph}{0pt}{0em}{0ex}

\titleformat{\subparagraph}[runin]{\normalfont\normalsize\bfseries}{\thesubparagraph}{0pt}{}{}
\titlespacing{\subparagraph}{0pt}{0em}{0ex}

% --- Списки ------------------------------------------------------------------
% Компактные списки без лишних вертикальных отступов; стиль нумерации — 1., 1.1.
\setdefaultenum{1.}{1.}{1.}{1.}
\setdefaultitem{--}{}{}{}
% При необходимости более плотных списков можно раскомментировать строку ниже.
%\setlength\itemsep{-1em}
\let\itemize\compactitem
\let\enditemize\endcompactitem
\let\enumerate\compactenum
\let\endenumerate\endcompactenum
\let\description\compactdesc
\let\enddescription\endcompactdesc
\pltopsep=\smallskipamount
\plitemsep=0pt
\plparsep=0pt

% Команда для отмены разрыва страниц перед списками
\makeatletter
\newcommand\mynobreakpar{\par\nobreak\@afterheading}
\makeatother

\renewcommand{\theenumi}{\arabic{enumi}}
\renewcommand{\theenumii}{\arabic{enumii}}
\renewcommand{\theenumiii}{\arabic{enumiii}}
\renewcommand{\theenumiv}{\arabic{enumiv}}

\renewcommand{\labelenumi}{\theenumi.}
\renewcommand{\labelenumii}{\theenumi.\theenumii.}
\renewcommand{\labelenumiii}{\theenumi.\theenumii.\theenumiii.}
\renewcommand{\labelenumiv}{\theenumi.\theenumii.\theenumiii.\theenumiv.}

% --- Сноски ------------------------------------------------------------------
% Высокое значение interfootnotelinepenalty предотвращает разрывы сносок.
\interfootnotelinepenalty=10000 % стараемся не рвать сноски на страницах

% --- Подписи к рисункам и таблицам ------------------------------------------
% Подписи выравниваются по ширине и используют единообразные параметры ГОСТ.
\captionsetup[table]{justification=justified}
\captionsetup[figure]{justification=justified,name=Рисунок,singlelinecheck=on,font=onehalfspacing}

% --- Список литературы -------------------------------------------------------
% Настройка стиля оформление ссылок + подключение пользовательских .bib файлов.
\usepackage[
  style=gost-numeric,
  sorting=none,
  language=auto,
  autolang=other
]{biblatex}
