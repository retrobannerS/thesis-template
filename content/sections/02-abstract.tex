\chapter*{Реферат}
\thispagestyle{plain}

Общий объем основного текста, без учета приложений~---
\pageref{end_of_main_text} страниц, с учетом приложений~---
\pageref{end_of_document}. Количество использованных источников~---
\hyperref[sec:bibliography]{\total{citenum}}. Количество приложений~---
\hyperref[sec:appendices]{\total{totalappendices}}.

Ключевые слова: нетехнические пользователи, базы данных, Text-to-SQL, NLIDB, XiYan-SQL, MCP.

Целью работы является разработка модели взаимодействия с ядром XiYan-SQL на основе MCP-клиента и
ее практическая реализация в виде прототипа веб-сервиса с эмуляцией ключевых компонентов.

В первом разделе проводится исследование проблематики, выявление современных подходов для ее решения,
проверяется актуальность задачи и производится выбор технологического ядра.

Второй раздел посвящен разработке модели взаимодействия технологического ядра XiYan-SQL с серверной частью
прототипа веб-сервиса.

В третьем разделе описывается процесс проектирования прототипа веб-сервиса,
выбору стека технологий для программной реализации, а также описанию самой программной реализации и
проведенного тестирования данной системы.

В приложении \ref{appendix-A} приведены графики лидеров в бенчмарках
для Text-to-SQL.

В приложении \ref{appendix-B} представлены листинги ключевых фрагментов исходного кода,
демонстрирующие реализацию основных модулей и функций спроектированного веб-сервиса.

В приложении \ref{appendix-C} выведен отчет фреймворка для тестирования \verb|pytest|,
отражающий проведенное тестирование прототипа веб-сервиса.