\chapter*{Заключение}
\addcontentsline{toc}{chapter}{Заключение}

В ходе выполнения учебно-исследовательской работы была решена задача разработки модели
MCP-клиента для взаимодействия с технологическим ядром XiYan-SQL.
Также реализован прототип веб-сервиса, предоставляющего естественно-языковой интерфейс к
базам данных. Были достигнуты следующие основные результаты:

\begin{compactitem}
      \item В аналитической части при обзоре академических источников
      была подтверждена проблема существования барьера для нетехнических пользователей при работе с базами данных и обоснована необходимость в разработке интуитивно понятных интерфейсов. Было показано, что естественно-языковые интерфейсы (NLIDB) являются наиболее перспективным направлением решения данной проблематики.
      \item Анализ современных систем Text-to-SQL выявил доминирование LLM-основанных фреймворков и позволил обосновать выбор технологического ядра XiYan-SQL как наиболее производительного и открытого решения. Проведённое исследование веб-сервисов подтвердило целесообразность разработки собственного открытого решения, ориентированного на свободный онлайн-доступ без необходимости использования сторонних API-ключей.
      \item Приведены аргументы в пользу необходимости разработки модели взаимодействия
      компонента-организатора на серверной части веб-сервиса с технологическим ядром.
      В качестве результата был приведен полный жизненный цикл
      обработки пользовательского запроса на естественном языке
      MCP-клиентом, с помощью которого производится интеграция XiYan-SQL.
      \item Спроектирована полная архитектура прототипа веб-сервиса с использованием языка моделирования UML. Построены диаграммы вариантов использования, компонентов и последовательности, детально описывающие статическую структуру и динамику взаимодействия элементов системы. Обоснован выбор технологического стека для tt программной реализации.
      \item Реализован программный прототип веб-сервиса, включающий серверную часть (Backend) с REST API, пользовательский интерфейс (Frontend) и, модуль МСР-клиента с эмуляцией вызовов к ядру XiYan-SQL. Описаны основные функции и модули, а также
      прототип пользовательского графического интерфейса.
      \item Проведено комплексное модульное тестирование системы с использованием фреймворка \verb|pytest|. Все 47 разработанных тестов, покрывающих основной функционал, успешно пройдены, что подтверждает корректность реализации спроектированной архитектуры.
\end{compactitem}

\paragraph{Предполагаемые направления для дальнейшей работы:}
\begin{compactitem}
      \item Замена эмулятора ядра NLIDB на полную интеграцию с реальным XiYan-SQL MCP Server для оценки производительности в реальных условиях.
      \item Поддержка многотабличных запросов с операциями JOIN и работа с более сложными реляционными базами данных.
      \item Расширение функционала для работы с различными диалектами SQL (например, PostgreSQL, MySQL) и NoSQL базами данных.
      \item Добавление поддержки запросов к базам данных на русском языке.
\end{compactitem}

Таким образом, в рамках данной работы был выполнен полный цикл от анализа предметной области до реализации и тестирования программного прототипа, который закладывает прочный фундамент для создания полноценного веб-сервиса.