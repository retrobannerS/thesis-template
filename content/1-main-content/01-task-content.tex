% ============================================================================
%  Assignment content: таблица с заданием и список литературы к работе.
% ============================================================================

% --- Подсказки по подписям ----------------------------------------------------
% Используйте макрос \signat для вставки подписи с датой в таблицу задания.
% Параметры:
%   - xsign, ysign — смещение подписи по горизонтали и вертикали
%   - scale — масштаб изображения подписи
%   - img — имя файла подписи (без расширения) из папки подписей, определенной в config/project-settings.tex 
%   - xdate, ydate — смещение даты относительно подписи
%   - date — текст даты
% Значения по умолчанию: xsign=-10pt, ysign=4pt, scale=0.3, img=supervisor,
% xdate=10pt, ydate=-10pt, date=01.01.2001.


% --- Таблица задания ---------------------------------------------------------
\projecttasks{
    % ------------------------------------------------------------
    % ------------------------------------------------------------
    % --- Аналитическая часть -------------------------------------------------
    \projecttask{\bfseries\projecttasknum}{\bfseries Аналитическая часть}{}{}{}
    \projecttask{\projectsubtasknum}
    {
        Выделить обязательные элементы курсовых работ для разных кафедр.
    }%
    {Текст РСПЗ}%
    {12.09.2025}
    {
        \signat[img=ManilovAV, xsign=0pt, ysign=0pt, scale=0.12, xdate=10pt, ydate=-20pt, date=12.09.2025]
    }
    % ------------------------------------------------------------
    \projecttask{\projectsubtasknum}
    {
        Собрать требования пользователей и определить сценарии использования шаблона.
    }%
    {Текст РСПЗ}%
    {19.09.2025}
    {
        \signat[img=ManilovAV, xsign=0pt, ysign=-10pt, scale=0.15, xdate=10pt, ydate=-42pt, date=20.09.2025]
    }
    % ------------------------------------------------------------
    \projecttask{\projectsubtasknum}
    {Составить список возможностей и автоматизированных функций шаблона, а также структуру каталога шаблона.}%
    {Текст РСПЗ}%
    {10.10.2025}
    {
        \signat[img=ManilovAV, xsign=0pt, ysign=-10pt, scale=0.15, xdate=10pt, ydate=-42pt, date=12.10.2025]
    }
    % ------------------------------------------------------------
    \projecttask{\projectsubtasknum}
    {\itshape
        Оформить расширенное содержание
        пояснительной записки (РСПЗ).
    }%
    {Текст РСПЗ}%
    {24.10.2025}
    {
        \signat[img=ManilovAV, xsign=0pt, ysign=0pt, scale=0.12, xdate=10pt, ydate=-20pt, date=24.10.2025]
    }

    % ------------------------------------------------------------
    % ------------------------------------------------------------
    % --- Теоретическая часть -------------------------------------------------
    \projecttask{\bfseries\projecttasknum}{\bfseries Теоретическая часть}{}{}{}
    \projecttask{\projectsubtasknum}
    {Описать внутреннюю структуру каталога шаблона.}%
    {Текст ПЗ}%
    {30.10.2025}
    {
        \signat[img=ManilovAV, xsign=0pt, ysign=0pt, scale=0.12, xdate=10pt, ydate=-20pt, date=01.11.2025]
    }
    % ------------------------------------------------------------
    \projecttask{\projectsubtasknum}
    {Задокументировать ключевые команды и окружения шаблона.}%
    {Текст ПЗ}%
    {04.11.2025}
    {
        \signat[img=ManilovAV, xsign=0pt, ysign=0pt, scale=0.12, xdate=10pt, ydate=-20pt, date=05.11.2025]
    }

    % ------------------------------------------------------------
    % ------------------------------------------------------------
    % --- Инженерная часть -----------------------------------------------------
    \projecttask{\bfseries\projecttasknum}{\bfseries Инженерная часть}{}{}{}
    \projecttask{\projectsubtasknum}
    {Описать назначение и структуру файлов папки \texttt{config}.}%
    {Текст ПЗ}%
    {11.11.2025}
    {
        \signat[img=ManilovAV, xsign=0pt, ysign=0pt, scale=0.12, xdate=10pt, ydate=-20pt, date=12.11.2025]
    }

    % ------------------------------------------------------------
    \projecttask{\projectsubtasknum}
    {Описать процесс включения в документ различных элементов.}%
    {Текст ПЗ}%
    {18.11.2025}
    {
        \signat[img=ManilovAV, xsign=0pt, ysign=0pt, scale=0.12, xdate=10pt, ydate=-20pt,  date=20.11.2025]
    }

    % ------------------------------------------------------------
    \projecttask{\projectsubtasknum}
    {Описать использование отсканированных титульных листов вместо сверстанных шаблонных титульников.}%
    {Текст ПЗ, примеры документов}%
    {25.11.2025}
    {
        \signat[img=ManilovAV, xsign=0pt, ysign=-10pt, scale=0.15, xdate=10pt, ydate=-42pt, date=27.11.2025]
    }

    % ------------------------------------------------------------
    \projecttask{\projectsubtasknum}
    {Описать процесс создания ссылок на различные элементы документа.}%
    {Текст ПЗ}%
    {02.12.2025}
    {
        \signat[img=ManilovAV, xsign=0pt, ysign=0pt, scale=0.12, xdate=10pt, ydate=-20pt, date=04.12.2025]
    }

    % ------------------------------------------------------------
    % ------------------------------------------------------------
    % --- Технологическая и практическая часть ---------------------------------
    \projecttask{\bfseries\projecttasknum}{\bfseries Технологическая и практическая часть}{}{}{}
    \projecttask{\projectsubtasknum}
    {Описать процесс сборки документов по шаблону: установку необходимого дистрибутива \LaTeX{} и подключаемых пакетов.}%
    {Текст ПЗ}%
    {09.12.2025}
    {
        \signat[img=ManilovAV, xsign=0pt, ysign=-10pt, scale=0.15, xdate=10pt, ydate=-62pt, date=11.12.2025]
    }
    % ------------------------------------------------------------
    \projecttask{\projectsubtasknum}
    {Описать использование VS Code и настройку расширения LaTeX~Workshop в IDE.}%
    {Текст ПЗ}%
    {16.12.2025}
    {
        \signat[img=ManilovAV, xsign=0pt, ysign=-10pt, scale=0.15, xdate=10pt, ydate=-42pt, date=16.12.2025]
    }
    % ------------------------------------------------------------
    \projecttask{\bfseries\projecttasknum}
    {\itshape
        Оформить пояснительную записку (ПЗ) и
        иллюстративный материал для доклада
    }%
    {Текст ПЗ, презентация}%
    {22.12.2025}
    {
        % Подпись закомментирована: задача еще не выполнена или подпись не требуется
        % \signat[img=ManilovAV, xsign=0pt, ysign=0pt, scale=0.12, xdate=10pt, ydate=-20pt, date=22.12.2025]
    }
}


% --- Список литературы -------------------------------------------------------
% Список литературы, которая должна быть включена в работу. Используется
% команда \nocite для включения источников в библиографию без явных ссылок
% в тексте задания.
\taskliterature{
    \nocite{
        kottwitzLaTeXBeginnersGuide2021,
        volnovISPOLZOVANIEIZDATELSKOYSISTEMY2022,
        mihaylovichKompyuternayaTipografiyaLaTeX2008
    }
}

