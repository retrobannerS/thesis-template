\chapter{Потом назовем}
\label{chapter1}

\begin{annotation}
      В данном разделе проводится анализ предметной области~--- подготовки и оформления исследовательских и квалификационных работ в соответствии с университетскими стандартами. Рассматриваются основные трудности, с которыми сталкиваются студенты при верстке научно-исследовательских работ, и определяются сценарии использования шаблона.

      Целью раздела является формирование четкого набора требований к функциональности и структуре разрабатываемого \LaTeX-шаблона, который должен автоматизировать рутинные процессы оформления и позволить автору сосредоточиться на содержании исследования.
\end{annotation}

\section{Описание требований кафедры №22 НИЯУ МИФИ по написанию исследовательских и квалификационных работ}
\begin{annotation}
      В данном подразделе приводятся требования написания исследовательских работ, указанные на сайте кафедры №22 НИЯУ МИФИ. В эти требования входят требования к оформлению документа, к его содержанию и другие методические указания.
\end{annotation}

В соответствии с информацией, представленной на официальном сайте кафедры №22 «Кибернетика» НИЯУ МИФИ (\url{https://kaf22.ru}), процесс выполнения и оформления учебно-исследовательских (УИР), научно-исследовательских работ (НИР) и выпускных квалификационных работ (ВКР) регламентируется рядом методических указаний.

Основные нормативные документы, определяющие требования к оформлению:
\sloppy
\emergencystretch=3em
\begin{itemize}
      \item Методические указания по написанию отчета (РСПЗ) по УИР/НИР: \url{https://kaf22.ru/wp-content/uploads/2020/06/metodicheskie_ukazaniya_k_napisaniyu_otcheta_8_nedeli.pdf};
      \item Методические указания по написанию пояснительной записки (ПЗ) к УИР/НИР: \url{https://kaf22.ru/wp-content/uploads/2020/06/metodicheskie_ukazaniya_k_napisaniyu_pz_k_uir_i_nir.pdf};
      \item Методические указания по написанию выпускной квалификационной работы (ВКР): \url{https://kaf22.ru/wp-content/uploads/2021/06/metodicheskie-ukazaniya-k-napisaniyu-vkr.pdf}.
\end{itemize}
\fussy
\emergencystretch=1pt

\subsection{Описание требований к оформлению документа}

Согласно методическим указаниям, текст работы должен соответствовать следующим общим требованиям к оформлению документа:

\begin{itemize}
      \item текст работы должен быть оформлен на листе бумаги формата А4;
      \item текст работы печатается черным цветом;
      \item текст работы должен быть набран шрифтом Times New Roman;
      \item параметры шрифта:
            \begin{itemize}
                  \item для основного текста~--- 12~пунктов;
                  \item для заголовков первого уровня (названия разделов)~--- размер шрифта 14~пунктов, начертание жирное;
                  \item для заголовко второго уровня (названия подразделов)~--- размер шрифта 13~пунктов, начертание жирное;
                  \item для заголовков третьего уровня (названия пунктов и подпунктов)~--- 12~пунктов;
            \end{itemize}
      \item межстрочный интервал: 1,5~см;
      \item отступ красной строки: 1,25~см;
      \item интервал между абзацами: 0~см;
      \item поля страницы: левое~--- 30~мм, правое~--- 10~мм, верхнее и нижнее~--- 20~мм;
      \item выравнивание основного текста: по ширине, за исключениями, оговоренными ниже.
\end{itemize}

Рисунки, таблицы и формулы должны соответствовать следующим требованиям:

\begin{itemize}
      \item все таблицы и рисунки обязательно должны иметь названия;
      \item название рисунка размещается под рисунком;
      \item параметры шрифта названия рисунка: размер шрифта 12 пунктов, выраванивание по центру;
      \item название таблицы размещается над таблицей;
      \item параметры шрифта названия таблицы: размер шрифта 12 пунктов, выравнивание слева без красной строки;
      \item текст внутри таблиц размещается с интервалом между строками в 1~см;
      \item допускается многоуровневая нумерация таблиц и рисунков;
      \item после номера таблицы или рисунка не ставится точка, например «Рисунок 1.1» или «Таблица 1.1.1»;
      \item между номером таблицы или рисунка и его названием ставится тире <Рисунок 1.1 – Типология архитектур ИЭС>>;
      \item если таблица занимает более одной страницы, то на второй и последующих страницах повторяется название таблицы, сверху указывается её номер в сочетании со словами <<Продолжение>> или <<Окончание>> (например, <<Продолжение таблицы 1.3>>);
      \item формулы размещаются по центру страницы;
      \item справа от формулы указывается номер формулы в круглых скобках.
\end{itemize}

Список литературы должен быть оформлен в соответствии с требованиями ГОСТ Р 7.0.5 2008 и нумероваться в порядке появления ссылок в тексте.

В методических указаниях кроме основных требований к оформлению документов, также приводятся требования к оформлению документов, напрямую не связанные с системой верстки, а относящиеся к оформлению документов с точки зрения содержания текста, например:

\begin{itemize}
      \item cтруктурировать текст не более чем на три уровня вложенности;
      \item писать разделы объемом более 10 страниц;
      \item выделять в рамках пункта подпункт только в случае, если подпункт занимает более 2 страниц;
      \item наличие пустых строк или вертикальных отступов между абзацами не допускается;
      \item правила вынесения рисунков, таблиц, кода в приложения;
      \item требования по заполнению страниц минимум на $\frac{1}{2}$;
      \item не следует заканчивать раздел или пункт в рамках раздела списком, рисунком или таблицей – только текстом.
\end{itemize}

\subsection{Описание требований к структурам различных документов к исследовательским работам}

В рамках работы над исследовательским проектом студент подготавливает последовательность документов, каждый из которых имеет регламентированную структуру.

\paragraph{Задание на исследовательскую работу}. Задание --- первичный документ объемом 1--2 страницы, содержащий титульную <<шапку>> работы, перечень задач к студенту в виде таблицы, список литературы, обязательной к изучению при выполнении работы, а также поле с датой утверждения задания и подписями руководителя и студента.

\paragraph{Расширенное содержание пояснительной записки (РСПЗ) к УИР/НИР}. РСПЗ представляет собой отчет о работе в середине семестра (контрольная точка «8-я неделя») для контроля выполнения УИР/НИР студентом. Объем документа обычно составляет 10--15 страниц.

Структура РСПЗ:

\begin{enumerate}
      \item Титульный лист.
      \item Задание (копия утвержденного документа).
      \item Реферат (1 страница).
      \item Введение (1--2 страницы).
      \item Полностью написанный Раздел 1 --- Аналитическая часть с приведенными аннотациями для каждого подраздела (минимум 5 строк).
      \item Расширенное содержание последующих разделов (2--5). Для каждого раздела приводится аннотация (минимум 10 строк), для каждого подраздела также приводится аннотация (минимум 5 строк).
      \item Заключение (ожидаемые результаты).
      \item Список использованной литературы.
\end{enumerate}

\subsubsection{Пояснительная записка (ПЗ) к УИР/НИР}
Итоговый документ (отчет) по работе. Состоит из формальных разделов (реферат, введение, заключение, список литературы) и основного содержания, разбитого на 4 смысловых раздела:
\begin{enumerate}
      \item \textbf{Аналитический раздел}: исследование предметной области, обзор аналогов, постановка задачи.
      \item \textbf{Теоретический раздел}: моделирование, математические модели, алгоритмы.
      \item \textbf{Инженерно-технологический раздел}: архитектура системы, выбор инструментов, проектирование.
      \item \textbf{Практический раздел}: программная реализация, тестирование, результаты.
\end{enumerate}

Важным требованием является наличие в конце каждого раздела подраздела «Выводы», содержащего 3--5 пунктов с кратким подведением итогов. Названия разделов должны быть содержательными и отражать суть работы (названия типа «Аналитическая часть» не допускаются).

\subsection{Количественные требования}

В методических указаниях приводятся жесткие количественные ограничения на объем работы и ее элементов (см. таблицу~\ref{tab:quantitative-requirements}).

\begin{table}[h!]
      \centering
      \caption{Количественные требования к отчетным документам}
      \label{tab:quantitative-requirements}
      \begin{tabular}{|p{0.5\linewidth}|c|c|}
            \hline
            \textbf{Требование}                       & \textbf{ПЗ к УИР} & \textbf{ПЗ к ВКР} \\
            \hline
            Общий объем текста (мин.), стр.           & 25                & 30                \\
            \hline
            Рекомендуемый общий объем, стр.           & 30--40            & 40--60            \\
            \hline
            Минимальный объем основных разделов, стр. & 20                & 25                \\
            \hline
            Количество основных разделов              & 3--4              & 3--4              \\
            \hline
            Количество выводов по каждому разделу     & 3--5              & 3--5              \\
            \hline
            Список литературы (мин. кол-во)           & 20                & 35                \\
            \hline
      \end{tabular}
\end{table}

Соблюдение данных требований проверяется нормоконтролером и влияет на итоговую оценку. Автоматизация проверки этих формальных критериев является одной из ключевых задач разрабатываемого шаблона.

\section{Анализ обязательных элементов учебных работ}
% Task: Выделить обязательные элементы курсовых работ для разных кафедр.

\section{Анализ требований пользователей и сценариев использования}
\label{sec:ch1-user-requirements}
% Task: Собрать требования пользователей и определить сценарии использования шаблона.

\section{Функциональные возможности и структура шаблона}
\label{sec:ch1-features}
% Task: Составить список возможностей и автоматизированных функций шаблона, а также структуру каталога шаблона.
