% ============================================================================
%  RSPZ document: основной файл для компиляции документа "Расширенное
%  содержание пояснительной записки".
% ============================================================================
% arara: xelatex
% arara: biber
% arara: xelatex: { synctex: true }

\documentclass[12pt,a4paper,oneside,final]{report}

% --- Подключение шаблона ----------------------------------------------------

% ============================================================================
%  Master preamble: подключает все модули в нужном порядке. Здесь нет логики,
%  только список \input, который можно расширять/упорядочивать по необходимости.
% ============================================================================

% ============================================================================
%  Packages module: подключает все используемые библиотеки, сгруппированные
%  по задачам (основные настройки документа, верстка таблиц/рисунков и т.д.).
%  Здесь нет логики конфигурации — только \usepackage.
% ============================================================================

% --- Базовые возможности документа ------------------------------------------
\usepackage{tabularx}
\usepackage{booktabs}
\usepackage{xltabular}

\usepackage{polyglossia}
\usepackage{csquotes}
\usepackage{xltxtra} % \XeLaTeX macro
\usepackage{fontspec}

\usepackage{fancyhdr}
\usepackage[a4paper,left=30mm,right=15mm,top=20mm,bottom=20mm,bindingoffset=0cm]{geometry}%

\usepackage{amsfonts}
\usepackage{amssymb}
\usepackage{amsmath}
\usepackage{amsthm}
\usepackage{comment}

\usepackage{calc}
\usepackage{ifthen}
\usepackage{graphicx}
\usepackage{subcaption}
\usepackage{pdfpages}
\usepackage{longtable}
\usepackage{multirow}
\usepackage{indentfirst}
\usepackage[unicode=true]{hyperref}
\usepackage{color}
\usepackage{pgf}
\usepackage{titling}
\usepackage{totcount}

% --- Работа со списками, заголовками и приложениями --------------------------
\usepackage{paralist}
\usepackage[singlelinecheck=false,labelsep=endash]{caption}
\usepackage{titlesec}
\usepackage{appendix}

% --- Дополнительные утилиты --------------------------------------------------
\usepackage{mathpartir}
\usepackage{tikz}
\usepackage{hhline}
\usepackage{listings}

% --- Локализация -------------------------------------------------------------
\usepackage{template/preamble/localizations}


% ============================================================================
%  Language & fonts module: настраивает языки документа и наборы шрифтов.
%  Все параметры сгруппированы, чтобы легко включать/отключать нужные опции.
% ============================================================================

% --- Базовые языки -----------------------------------------------------------
\setmainlanguage[numerals=cyrillic]{russian}
\setotherlanguages{english}

% --- Совместимость кодировок -------------------------------------------------
% Ниже — отключённые совместимые пакеты. Включайте их только если требуется
% поддержка нестандартных символов в старых документах.
%\usepackage{xunicode} % some extra unicode support
%\usepackage[utf8x]{inputenc}

\defaultfontfeatures{Ligatures=TeX}

% --- Наборы шрифтов ----------------------------------------------------------
% Настройка шрифтов для основного текста, без засечек и моноширинного.
% По умолчанию используются:
%   - Times New Roman      — основной текст (с засечками)
%   - Arial                — без засечек (например, заголовки)
%   - Courier New          — моноширинный текст (код, примеры)
%
% Чтобы изменить стиль документа — раскомментируйте пример ниже и укажите
% свои шрифты (установленные в системе). Например:
%
% \setmainfont{Liberation Serif}    % основной текст (с засечками)
% \setsansfont{Liberation Sans}     % без засечек
% \setmonofont{PT Mono}             % моноширинный
%
% Для кастомизации, замените имена шрифтов на желаемые, либо добавьте опции,
% например:
%   \setmainfont[Ligatures=TeX, BoldFont={Liberation Serif Bold}]{Liberation Serif}
%
% Документация: https://ctan.org/pkg/fontspec
% --------------------------------------------------------------------------

\newfontfamily\cyrillicfont{Times New Roman}
\newfontfamily\cyrillicfontsf{Arial}
\newfontfamily\cyrillicfonttt{Courier New}

\setmainfont{Times New Roman}
\setsansfont{Arial}
\setmonofont{Courier New}

\newfontfamily\englishfont[Script=Latin, Contextuals={WordInitial,WordFinal}]{Times New Roman}
\setotherlanguage[Numerals=Latin]{english}

% ============================================================================
%  Formatting module: здесь собраны все правила визуального оформления текста.
%  Структура файла повторяет основные сущности ГОСТ/университетских требований,
%  чтобы можно было быстро найти нужный блок и настроить его.
% ============================================================================

% --- Макет страницы и колонтитулы --------------------------------------------
% Управляет стилями страниц, включая выравнивание шапок и подвалов.
\makeatletter
\let\ps@plain\ps@fancy % Подчиняем первые страницы каждой главы общим правилам
\makeatother
\pagestyle{fancy}
\fancyhf{}
\fancyfoot[C]{\thepage}
\renewcommand{\headrulewidth}{0pt}
\renewcommand{\footrulewidth}{0pt}
\renewcommand{\baselinestretch}{1.5}
\newcommand{\headertext}[1]{\fancyhead[R]{\tiny{#1}}}

% --- Заголовки ---------------------------------------------------------------
% Заголовки отдельных уровней оформлены так, чтобы соответствовать нормам ПЗ:
% главы — центрированы и жирные, остальные уровни — выровнены по левому краю.
\titleformat{\chapter}[block]{\centering\normalfont\Large\bfseries}{\thechapter.}{1ex}{}{}
\titlespacing{\chapter}{0pt}{0em}{2em}

\titleformat{\section}[block]{\normalfont\large\bfseries}{\thesection}{1ex}{}{}
\titlespacing{\section}{0pt}{0em}{1ex}

\titleformat{\subsection}[block]{\normalfont\normalsize\bfseries}{\thesubsection}{1ex}{}{}
\titlespacing{\section}{0pt}{0em}{1ex}

% paragraph и subparagraph — в тексте, без отступов
\titleformat{\paragraph}[runin]{\normalfont\normalsize\bfseries}{\theparagraph}{0pt}{}{}
\titlespacing{\paragraph}{0pt}{0em}{0ex}

\titleformat{\subparagraph}[runin]{\normalfont\normalsize\bfseries}{\thesubparagraph}{0pt}{}{}
\titlespacing{\subparagraph}{0pt}{0em}{0ex}

% --- Списки ------------------------------------------------------------------
% Компактные списки без лишних вертикальных отступов; стиль нумерации — 1., 1.1.
\setdefaultenum{1.}{1.}{1.}{1.}
\setdefaultitem{--}{}{}{}
% При необходимости более плотных списков можно раскомментировать строку ниже.
%\setlength\itemsep{-1em}
\let\itemize\compactitem
\let\enditemize\endcompactitem
\let\enumerate\compactenum
\let\endenumerate\endcompactenum
\let\description\compactdesc
\let\enddescription\endcompactdesc
\pltopsep=\smallskipamount
\plitemsep=0pt
\plparsep=0pt

% Команда для отмены разрыва страниц перед списками
\makeatletter
\newcommand\mynobreakpar{\par\nobreak\@afterheading}
\makeatother

\renewcommand{\theenumi}{\arabic{enumi}}
\renewcommand{\theenumii}{\arabic{enumii}}
\renewcommand{\theenumiii}{\arabic{enumiii}}
\renewcommand{\theenumiv}{\arabic{enumiv}}

\renewcommand{\labelenumi}{\theenumi.}
\renewcommand{\labelenumii}{\theenumi.\theenumii.}
\renewcommand{\labelenumiii}{\theenumi.\theenumii.\theenumiii.}
\renewcommand{\labelenumiv}{\theenumi.\theenumii.\theenumiii.\theenumiv.}

% --- Сноски ------------------------------------------------------------------
% Высокое значение interfootnotelinepenalty предотвращает разрывы сносок.
\interfootnotelinepenalty=10000 % стараемся не рвать сноски на страницах

% --- Подписи к рисункам и таблицам ------------------------------------------
% Подписи выравниваются по ширине и используют единообразные параметры ГОСТ.
\captionsetup[table]{justification=justified}
\captionsetup[figure]{justification=justified,name=Рисунок,singlelinecheck=on,font=onehalfspacing}

% --- Список литературы -------------------------------------------------------
% Настройка стиля оформление ссылок + подключение пользовательских .bib файлов.
\usepackage[
  style=gost-numeric,
  sorting=none,
  language=auto,
  autolang=other
]{biblatex}

% ============================================================================
%  Automation module: сбор утилит, которые автоматически считают объёмы,
%  формируют подписи или другие рутинные элементы документа.
% ============================================================================

% --- Библиография ------------------------------------------------------------
% Счётчик количества источников для автоматического упоминания в реферате.
\newtotcounter{citenum}
\AtEveryBibitem{\stepcounter{citenum}}

% --- Приложения --------------------------------------------------------------
% Счётчик общего количества приложений для автоматического упоминания в реферате.
\newcounter{totalappendices}
\regtotcounter{totalappendices}

% --- Общие статистики --------------------------------------------------------
% Суммарные цифры по рисункам, таблицам и листингам для автоматического упоминания в реферате.
\newcounter{totalfigures}
\newcounter{totaltables}
\newcounter{totallistings}


% ============================================================================
%  Project macros module: содержит все пользовательские макросы с данными
%  проекта (типы работ, участники, утверждения и т.д.), а также служебные
%  счётчики для нумерации задач.
% ============================================================================

% --- Общие сведения о проекте -----------------------------------------------
\makeatletter
\newcommand*{\projecttypefulldative}[1]{\gdef\@projecttypefulldative{#1}}
\newcommand*{\theprojecttypefulldative}{\@projecttypefulldative}
\newcommand*{\projecttypeshort}[1]{\gdef\@projecttypeshort{#1}}
\newcommand*{\theprojecttypeshort}{\@projecttypeshort}
\newcommand*{\authorfulldative}[1]{\gdef\@authorfulldative{#1}}
\newcommand*{\theauthorfulldative}{\@authorfulldative}
\newcommand*{\authorgroup}[1]{\gdef\@authorgroup{#1}}
\newcommand*{\theauthorgroup}{\@authorgroup}

% --- Руководители и консультанты --------------------------------------------
\newcommand*{\supervisor}[1]{\gdef\@supervisor{#1}}
\newcommand*{\thesupervisor}{\@supervisor}
\newcommand*{\consultant}[1]{\gdef\@consultant{#1}}
\newcommand*{\theconsultant}{\@consultant}

% --- Задания и литература ----------------------------------------------------
\newcommand{\projecttasks}[1]{\gdef\@projecttasks{#1}}
\newcommand{\theprojecttasks}{\@projecttasks}
\newcommand{\projecttask}[5]{#1 & #2 & #3 & #4 & #5 \\\hline}
\newcommand*{\taskliterature}[1]{\gdef\@taskliterature{#1}}
\newcommand*{\thetaskliterature}{\@taskliterature}

% --- Сроки и утверждения по заданию -----------------------------------------
\newcommand*{\taskdate}[1]{\gdef\@taskdate{#1}}
\newcommand*{\thetaskdate}{\@taskdate}
\newcommand*{\supervisortaskapproval}[1]{\gdef\@supervisortaskapproval{#1}}
\newcommand*{\thesupervisortaskapproval}{\@supervisortaskapproval}
\newcommand*{\authortaskapproval}[1]{\gdef\@authortaskapproval{#1}}
\newcommand*{\theauthortaskapproval}{\@authortaskapproval}

% --- Утверждения для РСПЗ ----------------------------------------------------
\newcommand*{\authorrspzapproval}[1]{\gdef\@authorrspzapproval{#1}}
\newcommand*{\theauthorrspzapproval}{\@authorrspzapproval}
\newcommand*{\supervisorrspzapproval}[1]{\gdef\@supervisorrspzapproval{#1}}
\newcommand*{\thesupervisorrspzapproval}{\@supervisorrspzapproval}
\newcommand*{\consultantrspzapproval}[1]{\gdef\@consultantrspzapproval{#1}}
\newcommand*{\theconsultantrspzapproval}{\@consultantrspzapproval}
\newcommand*{\supervisorrspzgrade}[1]{\gdef\@supervisorrspzgrade{#1}}
\newcommand*{\thesupervisorrspzgrade}{\@supervisorrspzgrade}
\newcommand*{\consultantrspzgrade}[1]{\gdef\@consultantrspzgrade{#1}}
\newcommand*{\theconsultantrspzgrade}{\@consultantrspzgrade}

% --- Утверждения для ПЗ ------------------------------------------------------
\newcommand*{\authorpzapproval}[1]{\gdef\@authorpzapproval{#1}}
\newcommand*{\theauthorpzapproval}{\@authorpzapproval}
\newcommand*{\supervisorpzapproval}[1]{\gdef\@supervisorpzapproval{#1}}
\newcommand*{\thesupervisorpzapproval}{\@supervisorpzapproval}
\newcommand*{\consultantpzapproval}[1]{\gdef\@consultantpzapproval{#1}}
\newcommand*{\theconsultantpzapproval}{\@consultantpzapproval}
\newcommand*{\supervisorpzgrade}[1]{\gdef\@supervisorpzgrade{#1}}
\newcommand*{\thesupervisorpzgrade}{\@supervisorpzgrade}
\newcommand*{\consultantpzgrade}[1]{\gdef\@consultantpzgrade{#1}}
\newcommand*{\theconsultantpzgrade}{\@consultantpzgrade}
\makeatother

% --- Счётчики задач ----------------------------------------------------------
\newcounter{projecttasknumber}
\newcommand{\projecttasknum}{\setcounter{projectsubtasknumber}{0}\stepcounter{projecttasknumber}\theprojecttasknumber.}

\newcounter{projectsubtasknumber}
\newcommand{\projectsubtasknum}{\stepcounter{projectsubtasknumber}\theprojecttasknumber.\theprojectsubtasknumber.}

% --- Новые условные флаги ---------------------------------------------------
\newif\ifshowannotations
\newif\ifusepdfTaskSheet
\newif\ifusepdfRspzTitle
\newif\ifusepdfPzTitle

\showannotationsfalse
\usepdfTaskSheetfalse
\usepdfRspzTitlefalse
\usepdfPzTitlefalse
% ============================================================================
%  Signatures module: отвечает за возможность автоматического добавления подписей, дат и
%  пустых полей в формах согласования.
% ============================================================================

% --- Конфигурация параметров подписи ----------------------------------------
\makeatletter
\pgfkeys{
    /signat/.is family, /signat,
    xsign/.initial=-10pt,
    ysign/.initial=4pt,
    scale/.initial=0.3,
    img/.initial=supervisor,
    xdate/.initial=10pt,
    ydate/.initial=-10pt,
    date/.initial=01.01.2001
}
\makeatother

% --- Макросы для вставки подписи --------------------------------------------
\providecommand{\signaturepath}{assets/signatures/}
\newcommand{\sign}[3][0pt]{%
    \tikz[overlay]{\node[yshift=#1]{\includegraphics[scale=#2]{\signaturepath#3.png}}}%
}
\NewDocumentCommand{\signat}{O{}}{%
    \pgfkeys{/signat, #1}%
    \begin{tikzpicture}[overlay]
        \node[xshift=\pgfkeysvalueof{/signat/xsign}, yshift=\pgfkeysvalueof{/signat/ysign}](c)
        {\includegraphics[scale=\pgfkeysvalueof{/signat/scale}]{\signaturepath\pgfkeysvalueof{/signat/img}.png}};
        \node[xshift=\pgfkeysvalueof{/signat/xdate}, yshift=\pgfkeysvalueof{/signat/ydate}]
        {\small\textit{\pgfkeysvalueof{/signat/date}}};
    \end{tikzpicture}
}
\input{template/preamble/listings.tex}
% Подключение пользовательских данных и основных путей
% ============================================================================
%  Project settings: пользовательские данные, пути к ресурсам и библиография.
% ============================================================================

% --- Участники и тип работы -------------------------------------------------
% Здесь прописываются основные данные автора и научного руководителя.
\authorgroup{Б22-555}
\author{Петечкин В. П.}
\authorfulldative{Петечкину Василию Петровичу}
\supervisor{Манилов А. В.}
\consultant{\emptyfield}
\projecttypefulldative{учебно-исследовательской работе}
\projecttypeshort{УИР}

% --- Тема работы ------------------------------------------------------------
% Название работы (тема УИР/диплома).
\title{Разработка модели интеграции
    технологического ядра XiYan-SQL во внешний
    веб-сервис и его программная реализация}

% --- Пути к ресурсам --------------------------------------------------------
% Правило поиска изображений относительно корня проекта; добавляйте каталоги через {}. Например: 
%\graphicspath{{assets/img/}{assets/img/additional/}}
\graphicspath{{assets/img/}}
% Каталог с изображениями подписей; PNG без расширения и с одинаковым DPI.
\renewcommand{\signaturepath}{signatures/}

    
% --- Библиография -----------------------------------------------------------
% Подключайте один или несколько .bib файлов; порядок команд влияет на приоритет.
\addbibresource{references/bibliography.bib}
% \addbibresource{references/additional-sources.bib}
% \addbibresource{references/additional-sources-2.bib}
% ...
\input{config/task-data}


% --- Настройки для РСПЗ ------------------------------------------------------
% В РСПЗ мы показываем аннотации разделов и подразделов
\showannotationstrue

% --- Подключение содержимого задания ---------------------------------------
% Необходимо для включения листа задания в документ
% ============================================================================
%  Assignment content: таблица с заданием и список литературы к работе.
% ============================================================================

% --- Подсказки по подписям ----------------------------------------------------
% Используйте макрос \signat для вставки подписи с датой в таблицу задания.
% Параметры:
%   - xsign, ysign — смещение подписи по горизонтали и вертикали
%   - scale — масштаб изображения подписи
%   - img — имя файла подписи (без расширения) из папки подписей, определенной в config/project-settings.tex 
%   - xdate, ydate — смещение даты относительно подписи
%   - date — текст даты
% Значения по умолчанию: xsign=-10pt, ysign=4pt, scale=0.3, img=supervisor,
% xdate=10pt, ydate=-10pt, date=01.01.2001.


% --- Таблица задания ---------------------------------------------------------
\projecttasks{
    \projecttask{\bfseries\projecttasknum}{\bfseries Аналитическая часть}{}{}{}
    \projecttask{\projectsubtasknum}
    {\dots}%
    {Текст РСПЗ}%
    {\dots}
    {
        \signat[img=ManilovAV, xsign=-7pt, ysign=8pt, scale=0.10, xdate=14pt, ydate=-2pt, date={\footnotesize05.09.2025}]
    }
    % ------------------------------------------------------------
    \projecttask{\projectsubtasknum}
    {\dots}%
    {Текст РСПЗ}%
    {\dots}
    {
        \signat[img=ManilovAV, xsign=-7pt, ysign=8pt, scale=0.10, xdate=14pt, ydate=-2pt, date={\footnotesize12.09.2025}]
    }
    % ------------------------------------------------------------
    \projecttask{\projectsubtasknum}
    {\itshape
        Оформить расширенное содержание
        пояснительной записки (РСПЗ).
    }%
    {Текст РСПЗ}%
    {24.10.2025}
    {
        % Подпись закомментирована: задача еще не выполнена или подпись не требуется
        % \signat[xsign=-12pt, ysign=-4pt, scale=0.25, xdate=14pt, ydate=-10pt, date={\footnotesize01.01.2001}]
    }

    % ------------------------------------------------------------
    % ------------------------------------------------------------
    % --- Теоретическая часть -------------------------------------------------
    \projecttask{\bfseries\projecttasknum}{\bfseries Теоретическая часть}{}{}{}
    \projecttask{\projectsubtasknum}
    {\dots}%
    {\dots}%
    {\dots}
    {
        % Подпись закомментирована: задача еще не выполнена или подпись не требуется
        % \signat[xsign=-12pt, ysign=-4pt, scale=0.25, xdate=14pt, ydate=-10pt, date={\footnotesize01.01.2001}]
    }

    % ------------------------------------------------------------
    \projecttask{\projectsubtasknum}
    {\dots}%
    {Текст ПЗ}%
    {\dots}
    {
        % Подпись закомментирована: задача еще не выполнена или подпись не требуется
        % \signat[xsign=-12pt, ysign=-4pt, scale=0.25, xdate=14pt, ydate=-10pt, date={\footnotesize01.01.2001}]
    }

    % ------------------------------------------------------------
    \projecttask{\projectsubtasknum}
    {\dots}%
    {Текст ПЗ}%
    {\dots}
    {
        % Подпись закомментирована: задача еще не выполнена или подпись не требуется
        % \signat[xsign=-12pt, ysign=-4pt, scale=0.25, xdate=14pt, ydate=-10pt, date={\footnotesize01.01.2001}]
    }

    % ------------------------------------------------------------
    % ------------------------------------------------------------
    % --- Инженерная часть -----------------------------------------------------
    \projecttask{\bfseries\projecttasknum}{\bfseries Инженерная часть}{}{}{}
    \projecttask{\projectsubtasknum}
    {\dots}%
    {Текст ПЗ}%
    {\dots}
    {
        % Подпись закомментирована: задача еще не выполнена или подпись не требуется
        % \signat[xsign=-12pt, ysign=-4pt, scale=0.25, xdate=14pt, ydate=-10pt, date={\footnotesize01.01.2001}]
    }

    % ------------------------------------------------------------
    \projecttask{\projectsubtasknum}
    {\dots}%
    {\dots}%
    {\dots}
    {
        % Подпись закомментирована: задача еще не выполнена или подпись не требуется
        % \signat[xsign=-12pt, ysign=-4pt, scale=0.25, xdate=14pt, ydate=-10pt, date={\footnotesize01.01.2001}]
    }

    % ------------------------------------------------------------
    \projecttask{\projectsubtasknum}
    {\dots}%
    {\dots}%
    {\dots}
    {
        % Подпись закомментирована: задача еще не выполнена или подпись не требуется
        % \signat[xsign=-12pt, ysign=-4pt, scale=0.25, xdate=14pt, ydate=-10pt, date={\footnotesize01.01.2001}]
    }

    % ------------------------------------------------------------
    % ------------------------------------------------------------
    % --- Технологическая и практическая часть ---------------------------------
    \projecttask{\bfseries\projecttasknum}{\bfseries Технологическая и практическая часть}{}{}{}
    \projecttask{\bfseries\projecttasknum}
    {\itshape
        Оформить пояснительную записку (ПЗ) и
        иллюстративный материал для доклада
    }%
    {Текст ПЗ, презентация}%
    {22.12.2025}
    {
        % Подпись закомментирована: задача еще не выполнена или подпись не требуется
        % \signat[xsign=-12pt, ysign=-4pt, scale=0.25, xdate=14pt, ydate=-10pt, date={\footnotesize01.01.2001}]
    }
}


% --- Список литературы -------------------------------------------------------
% Список литературы, которая должна быть включена в работу. Используется
% команда \nocite для включения источников в библиографию без явных ссылок
% в тексте задания.
\taskliterature{
    \nocite{
        borodinZadacheSostavleniyaZaprosov2016,
        kimNaturalLanguageSQL2020,
        zhuLargeLanguageModel2024,
        huangExploringLandscapeTexttoSQL2025,
        gaoPreviewXiYanSQLMultiGenerator2025
    }
}



\begin{document}

% --- Титульный лист РСПЗ ----------------------------------------------------
% Условная вставка: либо готовый PDF, либо LaTeX-верстка
\ifusepdfRspzTitle
    \includepdf[pages={-}, offset=0mm -0mm]{\rspztitlepdf}
\else
    % ============================================================================
%  RSPZ title page: титульный лист документа "Расширенное содержание ПЗ".
% ============================================================================
% Содержит шапку, тему работы, информацию об участниках с подписями и оценками,
% а также год и место выполнения работы.

\thispagestyle{empty}
\begin{center}
  {\scriptsize
    \uppercase{Министерство науки и высшего образования российской федерации}\linebreak
    \uppercase{Федеральное государственное автономное образовательное
      учреждение высшего образования}
  }

  \textbf{Национальный исследовательский ядерный университет «МИФИ»}

  {\footnotesize
    \noindent\makebox[\linewidth]{\rule{\linewidth}{0.4pt}}
  }
\end{center}

\vskip 1em

\noindent
\begin{tabular}{@{}ll@{}}
  \raisebox{-0.5\height}{\includegraphics[width=0.2\linewidth]{template/assets/mephi.png}}
   &
  \begin{tabular}{@{}c@{}}
    \textbf{\large{}Институт интеллектуальных кибернетических систем} \\
    \uppercase{\textbf{\large{}Кафедра кибернетики (№ 22)}}
  \end{tabular}
\end{tabular}


\vfill

% --- Заголовок документа ----------------------------------------------------
\begin{center}
    Направление подготовки 09.03.04 Программная инженерия

    \vfill

    {\Large{\textbf{Расширенное содержание пояснительной записки}}}

    к \theprojecttypefulldative\space студента на тему:

    {\Large\thetitle}
\end{center}

\vfill

% --- Информация об участниках ----------------------------------------------
{\large

    \noindent
    \begin{tabularx}{\linewidth}{@{}l>{\centering}c>{\centering}c>{\centering}c>{\centering}p{2.5cm}c>{\centering}lX@{}}
        Группа              &  & \raggedright\theauthorgroup &  &                            &  &                & \\
        Студент             &  &                             &  & \theauthorrspzapproval     &  & \theauthor     & \\ \hhline{~~~~-~}
        Руководитель        &  & \thesupervisorrspzgrade     &  & \thesupervisorrspzapproval &  & \thesupervisor & \\ \hhline{~~-~-~}
        Научный консультант &  & \theconsultantrspzgrade     &  & \theconsultantrspzapproval &  & \theconsultant & \\ \hhline{~~-~-~}
    \end{tabularx}

    \vfill

    \vfill

    % --- Место и год выполнения работы ----------------------------------------
    \begin{center}
        \textbf{Москва \the\year}
    \end{center}

}

\fi
\newpage

% --- Лист задания -----------------------------------------------------------
% Условная вставка: либо готовый PDF, либо LaTeX-верстка
\ifusepdfTaskSheet
    \includepdf[pages={-}, offset=0mm -0mm]{\tasktitlepdf}
\else
    \begin{center}
  {\scriptsize
    \uppercase{Министерство науки и высшего образования российской федерации}\linebreak
    \uppercase{Федеральное государственное автономное образовательное
      учреждение высшего образования}
  }

  \textbf{Национальный исследовательский ядерный университет «МИФИ»}

  {\footnotesize
    \noindent\makebox[\linewidth]{\rule{\linewidth}{0.4pt}}
  }
\end{center}

\vskip 1em

\noindent
\begin{tabular}{@{}ll@{}}
  \raisebox{-0.5\height}{\includegraphics[width=0.2\linewidth]{template/assets/mephi.png}}
   &
  \begin{tabular}{@{}c@{}}
    \textbf{\large{}Институт интеллектуальных кибернетических систем} \\
    \uppercase{\textbf{\large{}Кафедра кибернетики (№ 22)}}
  \end{tabular}
\end{tabular}



\begin{center}
    {\Large{\textbf{Задание на \theprojecttypeshort}}}\\

    \large

    Студенту гр. \theauthorgroup{} \theauthorfulldative
\end{center}


\begin{center}
    \uppercase{\textbf{\large{}Тема \theprojecttypeshort}}\\

    {\Large\thetitle}\\

    \vskip 1em

    \uppercase{\textbf{\large{}Задание}}
\end{center}


\begin{xltabular}{\linewidth}{|p{0.7cm}|X|>{\footnotesize\centering\arraybackslash}p{2cm}|>{\footnotesize\centering\arraybackslash}p{2cm}|>{\centering\arraybackslash}p{2.2cm}|}
    \hline
    \multicolumn{1}{|>{\centering\arraybackslash}p{0.7cm}|}{№\par п/п}
    & \multicolumn{1}{c|}{Содержание работы}
    & {\normalsize Форма \par отчетности}
    & {\normalsize Срок \par исполнения}
    & Отметка о \par выполнении \\
    \hline
    \theprojecttasks
\end{xltabular}


\refsection
\thetaskliterature
\begin{center}
    \uppercase{\textbf{\large{}Литература}}
\end{center}
\printbibliography[heading=none]
\setcounter{citenum}{0}
\endrefsection


\vfill

\begin{table}[!h]
    \captionsetup{type=table,skip=0pt}
    {\noindent\linespread{2.0}
        \begin{tabularx}{\linewidth}{p{140pt}X>{\centering}XX}
            Дата выдачи задания: & Руководитель & \thesupervisortaskapproval & \thesupervisor \\ \hhline{~~-~}
            \thetaskdate         & Студент      & \theauthortaskapproval     & \theauthor     \\ \hhline{~~-~}
        \end{tabularx}
    }
\end{table}
\fi
\newpage

% --- Настройка нумерации страниц --------------------------------------------
\pagestyle{plain}
\pagenumbering{arabic}
\setcounter{page}{2}

% --- Основное содержимое документа -------------------------------------------
% Используется отдельная refsection для изоляции библиографии РСПЗ
\refsection

\clearpage

\chapter*{Реферат}
\thispagestyle{plain}

Общий объем основного текста, без учета приложений~--- \pageref{end_of_main_text} страниц
\ifshowannotations
      , с учетом приложений~--- \pageref{end_of_document}. 
\else
      .
\fi
Количество использованных источников~--- \hyperref[sec:bibliography]{\total{citenum}}. 
%Количество приложений~---\hyperref[sec:appendices]{\total{totalappendices}}.

Ключевые слова: LaTeX, шаблон, исследовательская работа, оформление, автоматизация, документация, студент.

Целью работы является

В первом разделе проводится исследование проблематики,

Второй раздел посвящен разработке модели

В третьем разделе описывается процесс проектирования прототипа веб-сервиса,

% В приложении \ref{appendix-A} 

\clearpage

\input{content/1-main-content/3-toc}

\clearpage

\input{content/1-main-content/4-intro}

\clearpage

\chapter{Потом назовем}
\label{chapter1}

\begin{annotation}
      В данном разделе проводится анализ предметной области~--- подготовки и оформления исследовательских и квалификационных работ в соответствии с университетскими стандартами. Рассматриваются основные трудности, с которыми сталкиваются студенты при верстке научно-исследовательских работ, и определяются сценарии использования шаблона.

      Целью раздела является формирование четкого набора требований к функциональности и структуре разрабатываемого \LaTeX-шаблона, который должен автоматизировать рутинные процессы оформления и позволить автору сосредоточиться на содержании исследования.
\end{annotation}

\section{Описание требований кафедры №22 НИЯУ МИФИ по написанию исследовательских и квалификационных работ}
\begin{annotation}
      В данном подразделе приводятся требования написания исследовательских работ, указанные на сайте кафедры №22 НИЯУ МИФИ. В эти требования входят требования к оформлению документа, к его содержанию и другие методические указания.
\end{annotation}

В соответствии с информацией, представленной на официальном сайте кафедры №22 «Кибернетика» НИЯУ МИФИ (\url{https://kaf22.ru}), процесс выполнения и оформления учебно-исследовательских (УИР), научно-исследовательских работ (НИР) и выпускных квалификационных работ (ВКР) регламентируется рядом методических указаний.

Основные нормативные документы, определяющие требования к оформлению:
\sloppy
\emergencystretch=3em
\begin{itemize}
      \item Методические указания по написанию отчета (РСПЗ) по УИР/НИР: \url{https://kaf22.ru/wp-content/uploads/2020/06/metodicheskie_ukazaniya_k_napisaniyu_otcheta_8_nedeli.pdf};
      \item Методические указания по написанию пояснительной записки (ПЗ) к УИР/НИР: \url{https://kaf22.ru/wp-content/uploads/2020/06/metodicheskie_ukazaniya_k_napisaniyu_pz_k_uir_i_nir.pdf};
      \item Методические указания по написанию выпускной квалификационной работы (ВКР): \url{https://kaf22.ru/wp-content/uploads/2021/06/metodicheskie-ukazaniya-k-napisaniyu-vkr.pdf}.
\end{itemize}
\fussy
\emergencystretch=1pt

\subsection{Описание требований к оформлению документа}

Согласно методическим указаниям, текст работы должен соответствовать следующим общим требованиям к оформлению документа:

\begin{itemize}
      \item текст работы должен быть оформлен на листе бумаги формата А4;
      \item текст работы печатается черным цветом;
      \item текст работы должен быть набран шрифтом Times New Roman;
      \item параметры шрифта:
            \begin{itemize}
                  \item для основного текста~--- 12~пунктов;
                  \item для заголовков первого уровня (названия разделов)~--- размер шрифта 14~пунктов, начертание жирное;
                  \item для заголовко второго уровня (названия подразделов)~--- размер шрифта 13~пунктов, начертание жирное;
                  \item для заголовков третьего уровня (названия пунктов и подпунктов)~--- 12~пунктов;
            \end{itemize}
      \item межстрочный интервал: 1,5~см;
      \item отступ красной строки: 1,25~см;
      \item интервал между абзацами: 0~см;
      \item поля страницы: левое~--- 30~мм, правое~--- 10~мм, верхнее и нижнее~--- 20~мм;
      \item выравнивание основного текста: по ширине, за исключениями, оговоренными ниже.
\end{itemize}

Рисунки, таблицы и формулы должны соответствовать следующим требованиям:

\begin{itemize}
      \item все таблицы и рисунки обязательно должны иметь названия;
      \item название рисунка размещается под рисунком;
      \item параметры шрифта названия рисунка: размер шрифта 12 пунктов, выраванивание по центру;
      \item название таблицы размещается над таблицей;
      \item параметры шрифта названия таблицы: размер шрифта 12 пунктов, выравнивание слева без красной строки;
      \item текст внутри таблиц размещается с интервалом между строками в 1~см;
      \item допускается многоуровневая нумерация таблиц и рисунков;
      \item после номера таблицы или рисунка не ставится точка, например «Рисунок 1.1» или «Таблица 1.1.1»;
      \item между номером таблицы или рисунка и его названием ставится тире <Рисунок 1.1 – Типология архитектур ИЭС>>;
      \item если таблица занимает более одной страницы, то на второй и последующих страницах повторяется название таблицы, сверху указывается её номер в сочетании со словами <<Продолжение>> или <<Окончание>> (например, <<Продолжение таблицы 1.3>>);
      \item формулы размещаются по центру страницы;
      \item справа от формулы указывается номер формулы в круглых скобках.
\end{itemize}

Список литературы должен быть оформлен в соответствии с требованиями ГОСТ Р 7.0.5 2008 и нумероваться в порядке появления ссылок в тексте.

В методических указаниях кроме основных требований к оформлению документов, также приводятся требования к оформлению документов, напрямую не связанные с системой верстки, а относящиеся к оформлению документов с точки зрения содержания текста, например:

\begin{itemize}
      \item cтруктурировать текст не более чем на три уровня вложенности;
      \item писать разделы объемом более 10 страниц;
      \item выделять в рамках пункта подпункт только в случае, если подпункт занимает более 2 страниц;
      \item наличие пустых строк или вертикальных отступов между абзацами не допускается;
      \item правила вынесения рисунков, таблиц, кода в приложения;
      \item требования по заполнению страниц минимум на $\frac{1}{2}$;
      \item не следует заканчивать раздел или пункт в рамках раздела списком, рисунком или таблицей – только текстом.
\end{itemize}

\subsection{Описание требований к структурам различных документов к исследовательским работам}

В рамках работы над исследовательским проектом студент подготавливает последовательность документов, каждый из которых имеет регламентированную структуру.

\paragraph{Задание на исследовательскую работу}. Задание --- первичный документ объемом 1--2 страницы, содержащий титульную <<шапку>> работы, перечень задач к студенту в виде таблицы, список литературы, обязательной к изучению при выполнении работы, а также поле с датой утверждения задания и подписями руководителя и студента.

\paragraph{Расширенное содержание пояснительной записки (РСПЗ) к УИР/НИР}. РСПЗ представляет собой отчет о работе в середине семестра (контрольная точка «8-я неделя») для контроля выполнения УИР/НИР студентом. Объем документа обычно составляет 10--15 страниц.

Структура РСПЗ:

\begin{enumerate}
      \item Титульный лист.
      \item Задание (копия утвержденного документа).
      \item Реферат (1 страница).
      \item Введение (1--2 страницы).
      \item Полностью написанный Раздел 1 --- Аналитическая часть с приведенными аннотациями для каждого подраздела (минимум 5 строк).
      \item Расширенное содержание последующих разделов (2--5). Для каждого раздела приводится аннотация (минимум 10 строк), для каждого подраздела также приводится аннотация (минимум 5 строк).
      \item Заключение (ожидаемые результаты).
      \item Список использованной литературы.
\end{enumerate}

\subsubsection{Пояснительная записка (ПЗ) к УИР/НИР}
Итоговый документ (отчет) по работе. Состоит из формальных разделов (реферат, введение, заключение, список литературы) и основного содержания, разбитого на 4 смысловых раздела:
\begin{enumerate}
      \item \textbf{Аналитический раздел}: исследование предметной области, обзор аналогов, постановка задачи.
      \item \textbf{Теоретический раздел}: моделирование, математические модели, алгоритмы.
      \item \textbf{Инженерно-технологический раздел}: архитектура системы, выбор инструментов, проектирование.
      \item \textbf{Практический раздел}: программная реализация, тестирование, результаты.
\end{enumerate}

Важным требованием является наличие в конце каждого раздела подраздела «Выводы», содержащего 3--5 пунктов с кратким подведением итогов. Названия разделов должны быть содержательными и отражать суть работы (названия типа «Аналитическая часть» не допускаются).

\subsection{Количественные требования}

В методических указаниях приводятся жесткие количественные ограничения на объем работы и ее элементов (см. таблицу~\ref{tab:quantitative-requirements}).

\begin{table}[h!]
      \centering
      \caption{Количественные требования к отчетным документам}
      \label{tab:quantitative-requirements}
      \begin{tabular}{|p{0.5\linewidth}|c|c|}
            \hline
            \textbf{Требование}                       & \textbf{ПЗ к УИР} & \textbf{ПЗ к ВКР} \\
            \hline
            Общий объем текста (мин.), стр.           & 25                & 30                \\
            \hline
            Рекомендуемый общий объем, стр.           & 30--40            & 40--60            \\
            \hline
            Минимальный объем основных разделов, стр. & 20                & 25                \\
            \hline
            Количество основных разделов              & 3--4              & 3--4              \\
            \hline
            Количество выводов по каждому разделу     & 3--5              & 3--5              \\
            \hline
            Список литературы (мин. кол-во)           & 20                & 35                \\
            \hline
      \end{tabular}
\end{table}

Соблюдение данных требований проверяется нормоконтролером и влияет на итоговую оценку. Автоматизация проверки этих формальных критериев является одной из ключевых задач разрабатываемого шаблона.

\section{Анализ обязательных элементов учебных работ}
% Task: Выделить обязательные элементы курсовых работ для разных кафедр.

\section{Анализ требований пользователей и сценариев использования}
\label{sec:ch1-user-requirements}
% Task: Собрать требования пользователей и определить сценарии использования шаблона.

\section{Функциональные возможности и структура шаблона}
\label{sec:ch1-features}
% Task: Составить список возможностей и автоматизированных функций шаблона, а также структуру каталога шаблона.


\clearpage

\chapter{Теоретическая часть}\label{chapter2}

\clearpage

\chapter{Проектирование и программная реализация прототипа веб-сервиса}
\label{chapter3}

\begin{annotation}
      В данной главе описывается процесс проектирования и программной реализации прототипа веб-сервиса с естественно-языковым интерфейсом. На основе сформулированных в предыдущей главе требований, с использованием языка моделирования UML, разрабатывается архитектура системы. Представляются диаграммы вариантов использования, компонентов и последовательности, которые детально описывают как статическую структуру, так и динамику взаимодействия элементов системы. В завершение приводится описание выбранного технологического стека и ключевых аспектов реализации прототипа.
\end{annotation}

\section{Проектирование архитектуры системы с использованием UML}

Проектирование является критически важным этапом разработки любого программного продукта. Для визуализации и формализации архитектуры будущего веб-сервиса был использован унифицированный язык моделирования UML (Unified Modeling Language). Были построены три диаграммы, описывающие систему с разных уровней абстракции.

\subsection{Диаграмма вариантов использования (Use Case Diagram)}

Диаграмма вариантов использования является наиболее высокоуровневым представлением системы. Она определяет границы системы, ее основных действующих лиц (акторов) и цели, которые эти акторы могут достигать при взаимодействии с системой.

Для нашего веб-сервиса был определен один актор~--- \textbf{Пользователь}. Его основные цели (варианты использования) представлены на рис.~\ref{fig:use-case-diagram}.

\begin{figure}[ht]
      \centering
      \includegraphics[width=0.7\textwidth]{created-diagrams/uml/use-case-diagram.png}
      \caption{Диаграмма вариантов использования}
      \label{fig:use-case-diagram}
\end{figure}

Как видно из схемы, Пользователь может выполнять три ключевые функции:
\begin{compactitem}
      \item \textbf{Управлять таблицами}. Этот вариант использования включает в себя загрузку
      новых таблиц (CSV/Excel), их переименование и удаление, а также добавление
      описаний для столбцов.
      \item \textbf{Задавать вопросы к данным}. Основная функция системы,
      позволяющая пользователю формулировать запросы на естественном языке к выбранной таблице.
      \item \textbf{Управлять профилем}. Включает в себя регистрацию, аутентификацию и
      изменение данных своего аккаунта (псевдоним, пароль, аватар).
\end{compactitem}

\subsection{Диаграмма компонентов (Component Diagram)}

Диаграмма компонентов описывает физическую структуру системы, показывая,
из каких крупных программных блоков она состоит и как они связаны между собой.
Эта диаграмма является основной архитектурной схемой нашего веб-сервиса
(см.~рис.~\ref{fig:component-diagram}).

\begin{figure}[ht]
      \centering
      \includegraphics[width=1\textwidth]{created-diagrams/uml/component-diagram.png}
      \caption{Диаграмма компонентов системы}
      \label{fig:component-diagram}
\end{figure}

Система состоит из четырех основных компонентов:
\begin{compactenum}
      \item \textbf{Клиентское приложение (Frontend):} Компонент, работающий в браузере пользователя.
      Реализован на HTML/CSS/JS. Отвечает за пользовательский интерфейс и взаимодействие с
      бэкендом через REST API.
      \item \textbf{Серверное приложение (Backend):} Основной компонент, реализованный на
      Python с использованием фреймворка FastAPI. Он предоставляет REST API,
      управляет бизнес-логикой и выступает в роли организатора для других компонентов.
      \item \textbf{Модуль хранения данных:} Отвечает за персистентное хранение информации.
      Он состоит из базы данных \textbf{SQLite} для метаинформации
      (пользователи, таблицы, описания столбцов) и \textbf{Файлового хранилища} для
      загруженных CSV/Excel файлов и аватаров пользователей.
      \item \textbf{Ядро NLIDB (Эмулятор):} В рамках прототипа~--- это компонент-эмулятор,
      имитирующий работу настоящего XiYan-SQL MCP Server. Он предоставляет внутренний API,
      который полностью соответствует спроектированному протоколу взаимодействия с целевой системой.
\end{compactenum}

\subsection{Диаграмма последовательности (Sequence Diagram)}

Диаграмма последовательности детализирует взаимодействие между компонентами во времени.
Она наглядно демонстрирует, какие вызовы и в какой последовательности происходят для
выполнения конкретного варианта использования. На рис.~\ref{fig:sequence-diagram} показана
последовательность действий для самого важного сценария~--- обработки запроса пользователя на
естественном языке.

\begin{figure}[ht]
      \centering
      \includegraphics[width=\textwidth]{created-diagrams/uml/sequence-diagram.png}
      \caption{Диаграмма последовательности для обработки ЕЯ-запроса}
      \label{fig:sequence-diagram}
\end{figure}

Процесс, изображенный на диаграмме, состоит из следующих шагов:
\begin{compactenum}
      \item Пользователь вводит вопрос в интерфейсе и нажимает кнопку <<Выполнить>>.
      \item \textbf{Frontend} отправляет асинхронный \verb|POST| запрос на эндпоинт
      \verb|/api/query| \textbf{Backend}-сервера, передавая текст вопроса и ID выбранной таблицы.
      \item \textbf{Backend (Организатор)} получает запрос и\
      сначала обращается к \textbf{Модулю данных}, чтобы получить метаинформацию о таблице
      (путь к файлу, описания столбцов) из базы SQLite.
      \item Получив метаданные, \textbf{Backend} формирует из них M-Schema и
      обращается к \textbf{Ядру NLIDB (Эмулятору)}, передавая ему вопрос и схему.
      \item \textbf{Эмулятор} для тестовых сценариев возвращает заранее заготовленный SQL-запрос.
      \item \textbf{Backend} получает SQL. Он снова обращается к \textbf{Модулю данных},
      но на этот раз для выполнения операции: он считывает нужный CSV-файл в объект pandas
      DataFrame и выполняет над ним полученный SQL-запрос с помощью библиотеки \texttt{pandasql}.
      \item \textbf{Модуль данных} возвращает результат выполнения~--- новый DataFrame.
      \item \textbf{Backend} сериализует результирующий DataFrame в формат JSON и
      отправляет его обратно на \textbf{Frontend} в теле успешного HTTP-ответа.
      \item \textbf{Frontend} получает данные и отрисовывает их в виде таблицы для
      Пользователя.
\end{compactenum}

Данные диаграммы полностью описывают спроектированную архитектуру и
служат основой для этапа программной реализации.




\section{Выбор стека технологий}

На основе спроектированной в предыдущем разделе архитектуры был выбран конкретный стек
технологий для программной реализации прототипа. Выбор каждого инструмента и фреймворка
обусловлен функциональными требованиями проекта, необходимостью разработки прототипа и
возможностями для дальнейшего масштабирования системы.

\paragraph{Серверная часть (Backend)}. Для реализации серверной части был выбран язык
программирования \textbf{Python} и асинхронный веб-фреймворк \textbf{FastAPI}.
Этот выбор обусловлен следующими причинами:
\begin{compactitem}
      \item \textbf{Экосистема Python для ИИ и анализа данных}. Ключевое ядро системы,
      XiYan-SQL, написано на Python. Использование Python для бэкенда является наиболее
      нативным и эффективным решением, так как позволяет избежать сложностей межъязыкового
      взаимодействия и напрямую интегрировать необходимые библиотеки.
      \item \textbf{Высокая производительность FastAPI}. FastAPI построен на базе Starlette и Pydantic,
      что обеспечивает ему производительность, сопоставимую с решениями на Go и Node.js.
      Его асинхронная природа идеально подходит для обработки запросов, связанных с
      длительными операциями, такими как обращение к LLM.
      \item \textbf{Автоматическая документация API}. FastAPI автоматически генерирует
      интерактивную документацию для API (Swagger UI и ReDoc), что значительно упрощает
      процесс разработки, тестирования и отладки взаимодействия между клиентской и серверной частями.
      \item \textbf{Работа с данными}. Для манипуляции данными из загружаемых
      CSV-файлов используется библиотека \textbf{pandas}~--- де-факто стандарт для
      анализа данных в Python. Для выполнения SQL-запросов над объектами DataFrame
      используется библиотека \textbf{pandasql}, что позволяет обрабатывать данные из файлов так,
      как если бы они находились в реляционной базе данных.
\end{compactitem}

\paragraph{Клиентская часть (Frontend)}.
Для реализации пользовательского интерфейса был выбран базовый стек из
\textbf{HTML5, CSS3 и нативного JavaScript (Vanilla JS)} по следующим причинам:
\begin{compactitem}
      \item \textbf{Фокус на основной задаче}. Основная сложность и новизна проекта лежат в
      серверной части и архитектуре взаимодействия с NLIDB-ядром. Использование стандартного
      стека без тяжелых фреймворков (таких как React или Vue) позволяет сконцентрировать усилия
      на реализации ключевой функциональности.
      \item \textbf{Отсутствие зависимостей и простота развертывания}. Данный подход не требует
      сложной системы сборки и зависимостей (Node.js, npm), что упрощает разработку и
      развертывание прототипа.
      \item \textbf{Гибкость}. Спроектированный REST API полностью отделяет логику бэкенда от
      представления. Это означает, что в будущем клиентская часть может быть легко заменена на
      более современный фреймворк без каких-либо изменений в серверной архитектуре.
\end{compactitem}

\paragraph{База данных (Database)}.
Для хранения метаинформации (данные пользователей, сведения о загруженных таблицах и
описания столбцов) была выбрана легковесная встраиваемая СУБД \textbf{SQLite}. Вот причины, по
которым была выбрана именно она:
\begin{compactitem}
      \item \textbf{Серверная независимость}. SQLite не требует отдельного серверного процесса.
      База данных представляет собой один файл, что идеально подходит для прототипа и упрощает
      его переносимость и настройку.
      \item \textbf{Нативная поддержка в Python}. SQLite встроена в стандартную библиотеку Python,
      что избавляет от необходимости устанавливать внешние драйверы. Для удобной работы с базой
      данных используется \textbf{SQLAlchemy}~--- популярный ORM (Object-Relational Mapper),
      который позволяет работать с таблицами как с Python-объектами.
\end{compactitem}

Таким образом, выбранный технологический стек представляет собой сбалансированное решение,
которое позволяет быстро и эффективно реализовать прототип, полностью соответствующий
поставленным задачам, и при этом закладывает прочный фундамент для будущего развития и
усложнения системы.




\section{Описание программной реализации прототипа веб-сервиса}

На основе спроектированной архитектуры и выбранного стека технологий был реализован
программный прототип веб-сервиса. В данном разделе приводится описание ключевых модулей и
функций системы. Фрагменты исходного кода, иллюстрирующие реализацию, вынесены в приложение
(см.~прил.~\ref{appendix-B}).




\subsection{Описание реализованных функций и модулей}

Программный код проекта организован в модульную структуру, соответствующую современным
практикам разработки на FastAPI. Корневая директория содержит главный файл приложения
\verb|main.py|, файлы конфигурации и директории с различными компонентами системы:
\verb|api|, \verb|core|, \verb|db|, \verb|features|, \verb|services|.

\paragraph{Точка входа и конфигурация приложения}. Файл \verb|main.py| является точкой
входа в приложение. В нем создается экземпляр класса \verb|FastAPI|, настраиваются
middleware-компоненты (включая \verb|CORSMiddleware| для обработки кросс-доменных запросов),
монтируются директории для статических файлов (\verb|/static|) и загрузок (\verb|/uploads|),
а также подключаются все API-маршруты из модуля \verb|api|. Важной частью является функция
с жизненным циклом (\verb|lifespan|), которая при старте сервера создает необходимые директории,
например, для аватаров пользователей.

\paragraph{Модуль аутентификации и управления пользователями}.
Данная функциональность реализована в директории \verb|features/users/|.
\begin{compactenum}
      \item \textbf{Модель данных (\texttt{models.py})}. Описана модель \verb|User| с
      использованием SQLAlchemy ORM. Она содержит поля для хранения псевдонима,
      хэшированного пароля, URL аватара и флагов статуса пользователя (активен,
      суперпользователь, стандартный аватар).
      \item \textbf{Операции с данными (\texttt{crud.py})}. Реализован класс \verb|CRUDUser|,
      который инкапсулирует всю логику прямого взаимодействия с базой данных:
      создание пользователя, поиск по имени, аутентификация, обновление. Для хэширования
      паролей используется библиотека \verb|passlib| с алгоритмом bcrypt, что обеспечивает
      безопасное хранение учетных данных.
      \item \textbf{API эндпоинты (\texttt{api.py})}. Определены все публичные маршруты для
      работы с пользователями:
      \begin{compactenum}
            \item \verb|/register|: Регистрация нового пользователя с валидацией длины и
            формата псевдонима и пароля.
            \item \verb|/login/access-token|: Аутентификация пользователя и
            выдача JWT-токена доступа, который используется для авторизации всех
            последующих запросов.
            \item \verb|/me|, \verb|/me/username|, \verb|/me/password|: Маршруты для получения и
            обновления данных текущего пользователя.
            \item \verb|/me/avatar|: Эндпоинты для загрузки и удаления пользовательского аватара.
      \end{compactenum}
      \item \textbf{Сервис генерации аватаров (\texttt{services/avatar\_service.py})}. Выделенный
      сервис, который при регистрации или удалении кастомного аватара генерирует стандартное
      изображение с первой буквой псевдонима пользователя на цветном фоне, созданном на
      основе хэша от имени. Для генерации используется библиотека \verb|Pillow|.
\end{compactenum}

\paragraph{Модуль управления таблицами данных}.
Аналогично пользователям, логика работы с таблицами вынесена в модуль \verb|features/tables/|.
\begin{compactenum}
      \item \textbf{Модель данных (\texttt{models.py})}. Описана модель \verb|Table|,
      связанная с моделью \verb|User| отношением <<один-ко-многим>>. Хранит имя таблицы,
      оригинальное имя файла, путь к файлу на сервере и ID владельца.
      \item \textbf{Сервис обработки таблиц (\texttt{services/table\_service.py})}. Этот
      сервисный слой содержит основную бизнес-логику. Функция \verb|process_and_save_table|
      отвечает за получение загруженного файла, его валидацию
      (проверка расширения на \verb|.csv| или \verb|.xlsx|), очистку имени, проверку на
      дубликаты, сохранение файла на диск с уникальным именем (с помощью \verb|uuid|) и,
      наконец, вызов CRUD-функции для создания записи в БД. Функция \verb|get_table_preview|
      использует библиотеку \verb|pandas| для чтения файла и возвращает первые 5 строк,
      названия столбцов и общее количество строк.
      \item \textbf{API эндпоинты (\texttt{api.py})}. Предоставляют интерфейс для фронтенда:
      загрузка файла (\verb|/upload|), получение списка таблиц пользователя (\verb|/|) и
      удаление таблицы (\verb|/{table_id}|). Все маршруты защищены и требуют наличия токена
      аутентификации.
\end{compactenum}

\paragraph{Модуль обработки NL-запросов (Организатор и Эмулятор)}.
Это центральный модуль, реализующий основную функцию системы. В текущем прототипе он
состоит из двух частей:
\begin{compactenum}
      \item \textbf{Организатор:} Его роль выполняет обработчик эндпоинта
      \verb|/api/query/|. Он получает от клиента текст вопроса и ID таблицы,
      извлекает из БД метаданные, формирует из них M-Schema и передает их в ядро NLIDB.
      \item \textbf{Ядро NLIDB (Эмулятор):} Реализовано в виде функции-заглушки
      \verb|convert_text_to_sql| в файле \verb|services/text_to_sql_service.py|.
      Эта функция имитирует работу настоящего ядра XiYan-SQL. Для демонстрационных целей в
      ней реализована простая логика, которая возвращает заранее заготовленные SQL-запросы
      для нескольких ключевых слов.
\end{compactenum}

Пример реализации эмулятора приведен в листинге~\ref{lst:nli-core-mock}. Такая архитектура
с эмуляцией позволяет полностью протестировать сквозное взаимодействие всех компонентов системы,
отложив сложную интеграцию с реальным MCP-сервером на следующие этапы разработки.

После получения SQL-запроса от эмулятора, Организатор использует библиотеку \verb|pandasql|
для его выполнения над DataFrame, полученным из пользовательского CSV-файла, и
возвращает результат на клиентскую часть.




\subsection{Описание графического пользовательского интерфейса}

Графический пользовательский интерфейс (GUI) является ключевой точкой взаимодействия пользователя с
системой. При его проектировании основной упор был сделан на минимализм, интуитивность и простоту,
чтобы пользователи без технической подготовки могли легко освоить все функции сервиса. Интерфейс
полностью русифицирован.


\paragraph{Аутентификация пользователя}. Первое, с чем сталкивается пользователь, — это система
аутентификации. Она состоит из двух экранов: регистрации и входа.

На экране регистрации пользователю предлагается ввести псевдоним и пароль. Система предоставляет
немедленную обратную связь: проверяет, не занят ли псевдоним, и отображает требования к сложности
пароля, подсвечивая выполненные и невыполненные условия(см.~рис.~\ref{fig:registration}). Это помогает
пользователю с первого раза создать корректные учетные данные.

После успешной регистрации пользователь попадает на экран входа, где для доступа к системе достаточно
ввести свой псевдоним и пароль. В случае неверного ввода данных система отображает понятное сообщение
об ошибке(см.~рис.~\ref{fig:login}).

\begin{figure}[ht]
      \centering
      \begin{subfigure}[b]{0.48\textwidth}
            \centering
            \includegraphics[width=\linewidth]{GUI/registration.png}
            \caption{Экран регистрации}
            \label{fig:registration}
      \end{subfigure}
      \hfill
      \begin{subfigure}[b]{0.48\textwidth}
            \centering
            \includegraphics[width=\linewidth]{GUI/login.png}
            \caption{Экран входа в систему}
            \label{fig:login}
      \end{subfigure}
      \caption{Снимки экрана интерфейса аутентификации пользователя}
      \label{fig:auth_screens}
\end{figure}


\paragraph{Основной рабочий интерфейс}. После входа пользователь попадает на главный экран~---
«Конструктор запросов» (см.~рис.~\ref{fig:queries}). Это центральная рабочая область, разделенная на
две части. Слева расположена панель «Мои таблицы», где отображается список всех загруженных
пользователем таблиц. Кнопка «+» позволяет инициировать процесс добавления новой таблицы.

\begin{figure}[ht]
      \centering
      \includegraphics[width=\textwidth]{GUI/queries.png}
      \caption{Основной интерфейс конструктора запросов}
      \label{fig:queries}
\end{figure}

При нажатии на кнопку добавления таблицы открывается модальное окно
(см.~рис.~\ref{fig:table-uploading}), через которое пользователь может загрузить файл в формате CSV
или Excel. После выбора файла система автоматически анализирует его и отображает предварительный
просмотр первых 10 строк, что позволяет пользователю убедиться в корректности данных перед их
окончательной загрузкой.

\begin{figure}[ht]
      \centering
      \includegraphics[width=\textwidth]{GUI/table-uploading.png}
      \caption{Модальное окно загрузки и предпросмотра новой таблицы}
      \label{fig:table-uploading}
\end{figure}

Основная часть экрана «Конструктор запросов» содержит большое текстовое поле для ввода запроса на
естественном языке. Ниже расположены кнопки для управления процессом: «Сгенерировать SQL», «Показать
SQL» и «Экспорт в CSV». В области «Результат» отображается таблица с данными, полученными после
выполнения запроса.

\paragraph{Настройки пользователя}. На странице настроек (см.~рис.~\ref{fig:settings}) пользователь
может управлять своим профилем.
Интерфейс разделен на три логических блока:
\begin{compactitem}
      \item \textbf{Смена аватара:} Пользователь может загрузить собственное изображение или удалить
      его, вернувшись к аватару по умолчанию, который генерируется автоматически. Модальное окно для
      смены аватара показано на рис.~\ref{fig:avatar-updating}.
      \item \textbf{Смена имени пользователя:} Поле для изменения псевдонима с мгновенной проверкой
      доступности нового имени.
      \item \textbf{Смена пароля:} Форма для безопасного обновления пароля с подтверждением старого и
      вводом нового пароля с соблюдением требований безопасности.
\end{compactitem}

\begin{figure}[ht]
      \centering
      \includegraphics[width=0.8\textwidth]{GUI/avatar-updating.png}
      \caption{Модальное окно смены аватара}
      \label{fig:avatar-updating}
\end{figure}

\begin{figure}[ht]
      \centering
      \includegraphics[width=0.8\textwidth]{GUI/settings.png}
      \caption{Страница настроек пользователя}
      \label{fig:settings}
\end{figure}

В целом, спроектированный графический интерфейс является логичным, неперегруженным и полностью
ориентированным на выполнение ключевых задач пользователя, что соответствует целям разработки данного
прототипа.




\section{Тестирование реализованного прототипа веб-сервиса}

Для проверки корректности работы, надежности и соответствия функциональным требованиям, изложенным в
предыдущих разделах, было проведено комплексное модульное тестирование разработанного прототипа.
Тестирование является неотъемлемым этапом жизненного цикла разработки, позволяющим выявить и устранить
потенциальные ошибки на ранней стадии и гарантировать стабильность системы.

\paragraph{Инструменты и среда тестирования}. В качестве основного фреймворка для написания и запуска
тестов был выбран \textbf{pytest}~--- современный и мощный инструмент для тестирования на Python. Для
взаимодействия с API-эндпоинтами приложения использовался \verb|TestClient| из состава фреймворка
FastAPI. Данный подход позволяет отправлять HTTP-запросы напрямую к приложению, не требуя запуска
реального веб-сервера, что значительно ускоряет выполнение тестов и упрощает их настройку.

Для обеспечения полной изоляции тестового окружения все тесты выполнялись на отдельной, временной базе
данных \textbf{SQLite}. Специальная фикстура в \verb|pytest| перед запуском каждого теста создавала
новую, чистую базу данных и удаляла ее после завершения. Это гарантирует, что тесты являются
независимыми друг от друга и результаты одного теста не влияют на последующие.

\paragraph{Структура и организация тестов}. Все автоматизированные тесты расположены в директории
\verb|/tests| и логически сгруппированы в модули, соответствующие структуре основного приложения
(\verb|test_users.py|, \verb|test_tables.py| и т.д.).

Ключевая логика подготовки тестового окружения вынесена в конфигурационный файл
\verb|tests/conftest.py|, где определены основные фикстуры:
\begin{compactitem}
      \item \verb|db|. Управляет жизненным циклом тестовой базы данных, обеспечивая ее создание и
      очистку.
      \item \verb|client|. Предоставляет экземпляр \verb|TestClient| для выполнения запросов от имени
      неаутентифицированного пользователя.
      \item \verb|authorized_client|. Ключевая фикстура для тестирования защищенных эндпоинтов. Она
      инкапсулирует логику создания временного пользователя через API, его последующей аутентификации
      и возвращает клиент с уже установленным токеном авторизации. Это позволяет значительно
      упростить код тестов для защищенных маршрутов.
\end{compactitem}

\paragraph{Тестовое покрытие}. Автоматизированные тесты охватывают все ключевые функциональные модули
разработанного прототипа:
\begin{compactitem}
      \item \textbf{Модуль аутентификации и управления пользователями:} Протестированы сценарии
      регистрации нового пользователя, входа в систему, получения и обновления данных профиля
      (включая смену пароля и аватара). Также покрыты случаи обработки ошибок, такие как попытка
      регистрации с уже существующим именем.
      \item \textbf{Модуль управления таблицами:} Протестированы все CRUD-операции
      (создание, чтение, обновление, удаление) для таблиц данных. Особое внимание уделено проверке
      прав доступа: тесты подтверждают, что пользователь не может получить доступ к таблицам,
      принадлежащим другому пользователю.
      \item \textbf{Внутренние сервисы:} Проведены тесты для отдельных сервисных функций в изоляции, в
      частности, для сервиса \verb|AvatarService|, отвечающего за генерацию аватаров по умолчанию.
\end{compactitem}

\paragraph{Результаты тестирования}. В общей сложности было разработано 47 автоматизированных тестов,
покрывающих описанный выше функционал. \textbf{Все 47 тестов успешно пройдены}, что подтверждается
выводом утилиты \verb|pytest| (см.~прил.~\ref{appendix-C}). В ходе выполнения тестов не было
зафиксировано никаких предупреждений (warnings).

Таким образом, комплексное тестирование подтверждает, что разработанный прототип веб-сервиса
функционирует корректно и в полном соответствии со спроектированной архитектурой и заявленными
функциональными требованиями.

\clearpage

\input{content/1-main-content/8-conclusion}

% --- Подсчет общих счетчиков для реферата -----------------------------------
% Сохраняем значения счетчиков перед библиографией для использования в реферате
\setcounter{totalfigures}{\the\value{totalfigures}+\the\value{figure}}
\setcounter{figure}{0}
\setcounter{totaltables}{\the\value{totaltables}+\the\value{table}}
\setcounter{table}{0}
\setcounter{totallistings}{\the\value{totallistings}+\the\value{lstlisting}}
\setcounter{lstlisting}{0}

% Создаем метки для ссылок на общее количество фигур, таблиц и листингов
\makeatletter
\edef\@currentlabel{\the\value{totalfigures}}
\label{figures}
\edef\@currentlabel{\the\value{totaltables}}
\label{tables}
\edef\@currentlabel{\the\value{totallistings}}
\label{listings}
\makeatother

% --- Список использованной литературы ----------------------------------------
\clearpage

\phantomsection
\label{sec:bibliography}
\addcontentsline{toc}{chapter}{\bibname}	% Добавляем список литературы в оглавление
\printbibliography

\endrefsection

\label{end_of_main_text}

\label{end_of_document}

\end{document}
