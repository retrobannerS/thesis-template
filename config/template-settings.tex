% ============================================================================
%  Template settings: пользовательские настройки для работы шаблона.
% ============================================================================

% --- Пути к ресурсам --------------------------------------------------------
% Правило поиска изображений относительно корня проекта; добавляйте каталоги через {}. Например: 
%\graphicspath{{assets/img/}{assets/img/additional/}}
\graphicspath{{content/0-assets/images/}}
% Каталог с изображениями подписей; PNG без расширения и с одинаковым DPI.
\renewcommand{\signaturepath}{config/assets/signatures/}

% --- Управление титульными листами ------------------------------------------
% Переключение между LaTeX-версткой и вставкой готового PDF.

% Раскомментируйте, если нужно использовать LaTeX-верстку титульных листов
\usepdfTaskSheetfalse
\usepdfRspzTitlefalse
\usepdfPzTitlefalse

% Раскомментируйте, если нужно использовать готовые PDF в качестве титульных листов
% \usepdfTaskSheettrue
% \usepdfRspzTitletrue
% \usepdfPzTitletrue

% Путь к готовым PDF титульных листов
\newcommand{\tasktitlepdf}{config/assets/titles/1-task-title.pdf}
\newcommand{\rspztitlepdf}{config/assets/titles/2-rspz-title.pdf}
\newcommand{\pztitlepdf}{config/assets/titles/3-pz-title.pdf}


% --- Библиография -----------------------------------------------------------
% Подключайте один или несколько .bib файлов; порядок команд влияет на приоритет.
\addbibresource{content/0-assets/references/bibliography.bib}
% \addbibresource{content/0-assets/references/additional-sources.bib}
% \addbibresource{content/0-assets/references/additional-sources-2.bib}
% ...