% ============================================================================
%  PZ document: основной файл для компиляции документа "Пояснительная записка".
% ============================================================================
% arara: xelatex
% arara: biber
% arara: xelatex: { synctex: true }

\documentclass[12pt,a4paper,oneside,final]{report}

% --- Подключение шаблона ----------------------------------------------------

% ============================================================================
%  Master preamble: подключает все модули в нужном порядке. Здесь нет логики,
%  только список \input, который можно расширять/упорядочивать по необходимости.
% ============================================================================

% ============================================================================
%  Packages module: подключает все используемые библиотеки, сгруппированные
%  по задачам (основные настройки документа, верстка таблиц/рисунков и т.д.).
%  Здесь нет логики конфигурации — только \usepackage.
% ============================================================================

% --- Базовые возможности документа ------------------------------------------
\usepackage{tabularx}
\usepackage{booktabs}
\usepackage{xltabular}

\usepackage{polyglossia}
\usepackage{csquotes}
\usepackage{xltxtra} % \XeLaTeX macro
\usepackage{fontspec}

\usepackage{fancyhdr}
\usepackage[a4paper,left=30mm,right=15mm,top=20mm,bottom=20mm,bindingoffset=0cm]{geometry}%

\usepackage{amsfonts}
\usepackage{amssymb}
\usepackage{amsmath}
\usepackage{amsthm}
\usepackage{comment}

\usepackage{calc}
\usepackage{ifthen}
\usepackage{graphicx}
\usepackage{subcaption}
\usepackage{pdfpages}
\usepackage{longtable}
\usepackage{multirow}
\usepackage{indentfirst}
\usepackage[unicode=true]{hyperref}
\usepackage{color}
\usepackage{pgf}
\usepackage{titling}
\usepackage{totcount}

% --- Работа со списками, заголовками и приложениями --------------------------
\usepackage{paralist}
\usepackage[singlelinecheck=false,labelsep=endash]{caption}
\usepackage{titlesec}
\usepackage{appendix}

% --- Дополнительные утилиты --------------------------------------------------
\usepackage{mathpartir}
\usepackage{tikz}
\usepackage{hhline}
\usepackage{listings}

% --- Локализация -------------------------------------------------------------
\usepackage{template/preamble/localizations}


% ============================================================================
%  Language & fonts module: настраивает языки документа и наборы шрифтов.
%  Все параметры сгруппированы, чтобы легко включать/отключать нужные опции.
% ============================================================================

% --- Базовые языки -----------------------------------------------------------
\setmainlanguage[numerals=cyrillic]{russian}
\setotherlanguages{english}

% --- Совместимость кодировок -------------------------------------------------
% Ниже — отключённые совместимые пакеты. Включайте их только если требуется
% поддержка нестандартных символов в старых документах.
%\usepackage{xunicode} % some extra unicode support
%\usepackage[utf8x]{inputenc}

\defaultfontfeatures{Ligatures=TeX}

% --- Наборы шрифтов ----------------------------------------------------------
% Настройка шрифтов для основного текста, без засечек и моноширинного.
% По умолчанию используются:
%   - Times New Roman      — основной текст (с засечками)
%   - Arial                — без засечек (например, заголовки)
%   - Courier New          — моноширинный текст (код, примеры)
%
% Чтобы изменить стиль документа — раскомментируйте пример ниже и укажите
% свои шрифты (установленные в системе). Например:
%
% \setmainfont{Liberation Serif}    % основной текст (с засечками)
% \setsansfont{Liberation Sans}     % без засечек
% \setmonofont{PT Mono}             % моноширинный
%
% Для кастомизации, замените имена шрифтов на желаемые, либо добавьте опции,
% например:
%   \setmainfont[Ligatures=TeX, BoldFont={Liberation Serif Bold}]{Liberation Serif}
%
% Документация: https://ctan.org/pkg/fontspec
% --------------------------------------------------------------------------

\newfontfamily\cyrillicfont{Times New Roman}
\newfontfamily\cyrillicfontsf{Arial}
\newfontfamily\cyrillicfonttt{Courier New}

\setmainfont{Times New Roman}
\setsansfont{Arial}
\setmonofont{Courier New}

\newfontfamily\englishfont[Script=Latin, Contextuals={WordInitial,WordFinal}]{Times New Roman}
\setotherlanguage[Numerals=Latin]{english}

% ============================================================================
%  Formatting module: здесь собраны все правила визуального оформления текста.
%  Структура файла повторяет основные сущности ГОСТ/университетских требований,
%  чтобы можно было быстро найти нужный блок и настроить его.
% ============================================================================

% --- Макет страницы и колонтитулы --------------------------------------------
% Управляет стилями страниц, включая выравнивание шапок и подвалов.
\makeatletter
\let\ps@plain\ps@fancy % Подчиняем первые страницы каждой главы общим правилам
\makeatother
\pagestyle{fancy}
\fancyhf{}
\fancyfoot[C]{\thepage}
\renewcommand{\headrulewidth}{0pt}
\renewcommand{\footrulewidth}{0pt}
\renewcommand{\baselinestretch}{1.5}
\newcommand{\headertext}[1]{\fancyhead[R]{\tiny{#1}}}

% --- Заголовки ---------------------------------------------------------------
% Заголовки отдельных уровней оформлены так, чтобы соответствовать нормам ПЗ:
% главы — центрированы и жирные, остальные уровни — выровнены по левому краю.
\titleformat{\chapter}[block]{\centering\normalfont\Large\bfseries}{\thechapter.}{1ex}{}{}
\titlespacing{\chapter}{0pt}{0em}{2em}

\titleformat{\section}[block]{\normalfont\large\bfseries}{\thesection}{1ex}{}{}
\titlespacing{\section}{0pt}{0em}{1ex}

\titleformat{\subsection}[block]{\normalfont\normalsize\bfseries}{\thesubsection}{1ex}{}{}
\titlespacing{\section}{0pt}{0em}{1ex}

% paragraph и subparagraph — в тексте, без отступов
\titleformat{\paragraph}[runin]{\normalfont\normalsize\bfseries}{\theparagraph}{0pt}{}{}
\titlespacing{\paragraph}{0pt}{0em}{0ex}

\titleformat{\subparagraph}[runin]{\normalfont\normalsize\bfseries}{\thesubparagraph}{0pt}{}{}
\titlespacing{\subparagraph}{0pt}{0em}{0ex}

% --- Списки ------------------------------------------------------------------
% Компактные списки без лишних вертикальных отступов; стиль нумерации — 1., 1.1.
\setdefaultenum{1.}{1.}{1.}{1.}
\setdefaultitem{--}{}{}{}
% При необходимости более плотных списков можно раскомментировать строку ниже.
%\setlength\itemsep{-1em}
\let\itemize\compactitem
\let\enditemize\endcompactitem
\let\enumerate\compactenum
\let\endenumerate\endcompactenum
\let\description\compactdesc
\let\enddescription\endcompactdesc
\pltopsep=\smallskipamount
\plitemsep=0pt
\plparsep=0pt

% Команда для отмены разрыва страниц перед списками
\makeatletter
\newcommand\mynobreakpar{\par\nobreak\@afterheading}
\makeatother

\renewcommand{\theenumi}{\arabic{enumi}}
\renewcommand{\theenumii}{\arabic{enumii}}
\renewcommand{\theenumiii}{\arabic{enumiii}}
\renewcommand{\theenumiv}{\arabic{enumiv}}

\renewcommand{\labelenumi}{\theenumi.}
\renewcommand{\labelenumii}{\theenumi.\theenumii.}
\renewcommand{\labelenumiii}{\theenumi.\theenumii.\theenumiii.}
\renewcommand{\labelenumiv}{\theenumi.\theenumii.\theenumiii.\theenumiv.}

% --- Сноски ------------------------------------------------------------------
% Высокое значение interfootnotelinepenalty предотвращает разрывы сносок.
\interfootnotelinepenalty=10000 % стараемся не рвать сноски на страницах

% --- Подписи к рисункам и таблицам ------------------------------------------
% Подписи выравниваются по ширине и используют единообразные параметры ГОСТ.
\captionsetup[table]{justification=justified}
\captionsetup[figure]{justification=justified,name=Рисунок,singlelinecheck=on,font=onehalfspacing}

% --- Список литературы -------------------------------------------------------
% Настройка стиля оформление ссылок + подключение пользовательских .bib файлов.
\usepackage[
  style=gost-numeric,
  sorting=none,
  language=auto,
  autolang=other
]{biblatex}

% ============================================================================
%  Automation module: сбор утилит, которые автоматически считают объёмы,
%  формируют подписи или другие рутинные элементы документа.
% ============================================================================

% --- Библиография ------------------------------------------------------------
% Счётчик количества источников для автоматического упоминания в реферате.
\newtotcounter{citenum}
\AtEveryBibitem{\stepcounter{citenum}}

% --- Приложения --------------------------------------------------------------
% Счётчик общего количества приложений для автоматического упоминания в реферате.
\newcounter{totalappendices}
\regtotcounter{totalappendices}

% --- Общие статистики --------------------------------------------------------
% Суммарные цифры по рисункам, таблицам и листингам для автоматического упоминания в реферате.
\newcounter{totalfigures}
\newcounter{totaltables}
\newcounter{totallistings}


% ============================================================================
%  Project macros module: содержит все пользовательские макросы с данными
%  проекта (типы работ, участники, утверждения и т.д.), а также служебные
%  счётчики для нумерации задач.
% ============================================================================

% --- Общие сведения о проекте -----------------------------------------------
\makeatletter
\newcommand*{\projecttypefulldative}[1]{\gdef\@projecttypefulldative{#1}}
\newcommand*{\theprojecttypefulldative}{\@projecttypefulldative}
\newcommand*{\projecttypeshort}[1]{\gdef\@projecttypeshort{#1}}
\newcommand*{\theprojecttypeshort}{\@projecttypeshort}
\newcommand*{\authorfulldative}[1]{\gdef\@authorfulldative{#1}}
\newcommand*{\theauthorfulldative}{\@authorfulldative}
\newcommand*{\authorgroup}[1]{\gdef\@authorgroup{#1}}
\newcommand*{\theauthorgroup}{\@authorgroup}

% --- Руководители и консультанты --------------------------------------------
\newcommand*{\supervisor}[1]{\gdef\@supervisor{#1}}
\newcommand*{\thesupervisor}{\@supervisor}
\newcommand*{\consultant}[1]{\gdef\@consultant{#1}}
\newcommand*{\theconsultant}{\@consultant}

% --- Задания и литература ----------------------------------------------------
\newcommand{\projecttasks}[1]{\gdef\@projecttasks{#1}}
\newcommand{\theprojecttasks}{\@projecttasks}
\newcommand{\projecttask}[5]{#1 & #2 & #3 & #4 & #5 \\\hline}
\newcommand*{\taskliterature}[1]{\gdef\@taskliterature{#1}}
\newcommand*{\thetaskliterature}{\@taskliterature}

% --- Сроки и утверждения по заданию -----------------------------------------
\newcommand*{\taskdate}[1]{\gdef\@taskdate{#1}}
\newcommand*{\thetaskdate}{\@taskdate}
\newcommand*{\supervisortaskapproval}[1]{\gdef\@supervisortaskapproval{#1}}
\newcommand*{\thesupervisortaskapproval}{\@supervisortaskapproval}
\newcommand*{\authortaskapproval}[1]{\gdef\@authortaskapproval{#1}}
\newcommand*{\theauthortaskapproval}{\@authortaskapproval}

% --- Утверждения для РСПЗ ----------------------------------------------------
\newcommand*{\authorrspzapproval}[1]{\gdef\@authorrspzapproval{#1}}
\newcommand*{\theauthorrspzapproval}{\@authorrspzapproval}
\newcommand*{\supervisorrspzapproval}[1]{\gdef\@supervisorrspzapproval{#1}}
\newcommand*{\thesupervisorrspzapproval}{\@supervisorrspzapproval}
\newcommand*{\consultantrspzapproval}[1]{\gdef\@consultantrspzapproval{#1}}
\newcommand*{\theconsultantrspzapproval}{\@consultantrspzapproval}
\newcommand*{\supervisorrspzgrade}[1]{\gdef\@supervisorrspzgrade{#1}}
\newcommand*{\thesupervisorrspzgrade}{\@supervisorrspzgrade}
\newcommand*{\consultantrspzgrade}[1]{\gdef\@consultantrspzgrade{#1}}
\newcommand*{\theconsultantrspzgrade}{\@consultantrspzgrade}

% --- Утверждения для ПЗ ------------------------------------------------------
\newcommand*{\authorpzapproval}[1]{\gdef\@authorpzapproval{#1}}
\newcommand*{\theauthorpzapproval}{\@authorpzapproval}
\newcommand*{\supervisorpzapproval}[1]{\gdef\@supervisorpzapproval{#1}}
\newcommand*{\thesupervisorpzapproval}{\@supervisorpzapproval}
\newcommand*{\consultantpzapproval}[1]{\gdef\@consultantpzapproval{#1}}
\newcommand*{\theconsultantpzapproval}{\@consultantpzapproval}
\newcommand*{\supervisorpzgrade}[1]{\gdef\@supervisorpzgrade{#1}}
\newcommand*{\thesupervisorpzgrade}{\@supervisorpzgrade}
\newcommand*{\consultantpzgrade}[1]{\gdef\@consultantpzgrade{#1}}
\newcommand*{\theconsultantpzgrade}{\@consultantpzgrade}
\makeatother

% --- Счётчики задач ----------------------------------------------------------
\newcounter{projecttasknumber}
\newcommand{\projecttasknum}{\setcounter{projectsubtasknumber}{0}\stepcounter{projecttasknumber}\theprojecttasknumber.}

\newcounter{projectsubtasknumber}
\newcommand{\projectsubtasknum}{\stepcounter{projectsubtasknumber}\theprojecttasknumber.\theprojectsubtasknumber.}

% --- Новые условные флаги ---------------------------------------------------
\newif\ifshowannotations
\newif\ifusepdfTaskSheet
\newif\ifusepdfRspzTitle
\newif\ifusepdfPzTitle

\showannotationsfalse
\usepdfTaskSheetfalse
\usepdfRspzTitlefalse
\usepdfPzTitlefalse
% ============================================================================
%  Signatures module: отвечает за возможность автоматического добавления подписей, дат и
%  пустых полей в формах согласования.
% ============================================================================

% --- Конфигурация параметров подписи ----------------------------------------
\makeatletter
\pgfkeys{
    /signat/.is family, /signat,
    xsign/.initial=-10pt,
    ysign/.initial=4pt,
    scale/.initial=0.3,
    img/.initial=supervisor,
    xdate/.initial=10pt,
    ydate/.initial=-10pt,
    date/.initial=01.01.2001
}
\makeatother

% --- Макросы для вставки подписи --------------------------------------------
\providecommand{\signaturepath}{assets/signatures/}
\newcommand{\sign}[3][0pt]{%
    \tikz[overlay]{\node[yshift=#1]{\includegraphics[scale=#2]{\signaturepath#3.png}}}%
}
\NewDocumentCommand{\signat}{O{}}{%
    \pgfkeys{/signat, #1}%
    \begin{tikzpicture}[overlay]
        \node[xshift=\pgfkeysvalueof{/signat/xsign}, yshift=\pgfkeysvalueof{/signat/ysign}](c)
        {\includegraphics[scale=\pgfkeysvalueof{/signat/scale}]{\signaturepath\pgfkeysvalueof{/signat/img}.png}};
        \node[xshift=\pgfkeysvalueof{/signat/xdate}, yshift=\pgfkeysvalueof{/signat/ydate}]
        {\small\textit{\pgfkeysvalueof{/signat/date}}};
    \end{tikzpicture}
}
\input{template/preamble/listings.tex}
% Подключение пользовательских данных и основных путей
% ============================================================================
%  Project settings: пользовательские данные, пути к ресурсам и библиография.
% ============================================================================

% --- Участники и тип работы -------------------------------------------------
% Здесь прописываются основные данные автора и научного руководителя.
\authorgroup{Б22-555}
\author{Петечкин В. П.}
\authorfulldative{Петечкину Василию Петровичу}
\supervisor{Манилов А. В.}
\consultant{\emptyfield}
\projecttypefulldative{учебно-исследовательской работе}
\projecttypeshort{УИР}

% --- Тема работы ------------------------------------------------------------
% Название работы (тема УИР/диплома).
\title{Разработка модели интеграции
    технологического ядра XiYan-SQL во внешний
    веб-сервис и его программная реализация}

% --- Пути к ресурсам --------------------------------------------------------
% Правило поиска изображений относительно корня проекта; добавляйте каталоги через {}. Например: 
%\graphicspath{{assets/img/}{assets/img/additional/}}
\graphicspath{{assets/img/}}
% Каталог с изображениями подписей; PNG без расширения и с одинаковым DPI.
\renewcommand{\signaturepath}{signatures/}

    
% --- Библиография -----------------------------------------------------------
% Подключайте один или несколько .bib файлов; порядок команд влияет на приоритет.
\addbibresource{references/bibliography.bib}
% \addbibresource{references/additional-sources.bib}
% \addbibresource{references/additional-sources-2.bib}
% ...
\input{config/task-data}


% --- Настройки для ПЗ --------------------------------------------------------
% В ПЗ аннотации разделов и подразделов не показываются (в отличие от РСПЗ)
\showannotationsfalse

% --- Подключение содержимого задания ---------------------------------------
% Необходимо для включения листа задания в документ
% ============================================================================
%  Assignment content: таблица с заданием и список литературы к работе.
% ============================================================================

% --- Подсказки по подписям ----------------------------------------------------
% Используйте макрос \signat для вставки подписи с датой в таблицу задания.
% Параметры:
%   - xsign, ysign — смещение подписи по горизонтали и вертикали
%   - scale — масштаб изображения подписи
%   - img — имя файла подписи (без расширения) из папки подписей, определенной в config/project-settings.tex 
%   - xdate, ydate — смещение даты относительно подписи
%   - date — текст даты
% Значения по умолчанию: xsign=-10pt, ysign=4pt, scale=0.3, img=supervisor,
% xdate=10pt, ydate=-10pt, date=01.01.2001.


% --- Таблица задания ---------------------------------------------------------
\projecttasks{
    % ------------------------------------------------------------
    % ------------------------------------------------------------
    % --- Аналитическая часть -------------------------------------------------
    \projecttask{\bfseries\projecttasknum}{\bfseries Аналитическая часть}{}{}{}
    \projecttask{\projectsubtasknum}
    {
        Выделить обязательные элементы курсовых работ для разных кафедр.
    }%
    {Текст РСПЗ}%
    {12.09.2025}
    {
        \signat[img=ManilovAV, xsign=0pt, ysign=0pt, scale=0.12, xdate=10pt, ydate=-20pt, date=12.09.2025]
    }
    % ------------------------------------------------------------
    \projecttask{\projectsubtasknum}
    {
        Собрать требования пользователей и определить сценарии использования шаблона.
    }%
    {Текст РСПЗ}%
    {19.09.2025}
    {
        \signat[img=ManilovAV, xsign=0pt, ysign=-10pt, scale=0.15, xdate=10pt, ydate=-42pt, date=20.09.2025]
    }
    % ------------------------------------------------------------
    \projecttask{\projectsubtasknum}
    {Составить список возможностей и автоматизированных функций шаблона, а также структуру каталога шаблона.}%
    {Текст РСПЗ}%
    {10.10.2025}
    {
        \signat[img=ManilovAV, xsign=0pt, ysign=-10pt, scale=0.15, xdate=10pt, ydate=-42pt, date=12.10.2025]
    }
    % ------------------------------------------------------------
    \projecttask{\projectsubtasknum}
    {\itshape
        Оформить расширенное содержание
        пояснительной записки (РСПЗ).
    }%
    {Текст РСПЗ}%
    {24.10.2025}
    {
        \signat[img=ManilovAV, xsign=0pt, ysign=0pt, scale=0.12, xdate=10pt, ydate=-20pt, date=24.10.2025]
    }

    % ------------------------------------------------------------
    % ------------------------------------------------------------
    % --- Теоретическая часть -------------------------------------------------
    \projecttask{\bfseries\projecttasknum}{\bfseries Теоретическая часть}{}{}{}
    \projecttask{\projectsubtasknum}
    {Описать внутреннюю структуру каталога шаблона.}%
    {Текст ПЗ}%
    {30.10.2025}
    {
        \signat[img=ManilovAV, xsign=0pt, ysign=0pt, scale=0.12, xdate=10pt, ydate=-20pt, date=01.11.2025]
    }
    % ------------------------------------------------------------
    \projecttask{\projectsubtasknum}
    {Задокументировать ключевые команды и окружения шаблона.}%
    {Текст ПЗ}%
    {04.11.2025}
    {
        \signat[img=ManilovAV, xsign=0pt, ysign=0pt, scale=0.12, xdate=10pt, ydate=-20pt, date=05.11.2025]
    }

    % ------------------------------------------------------------
    % ------------------------------------------------------------
    % --- Инженерная часть -----------------------------------------------------
    \projecttask{\bfseries\projecttasknum}{\bfseries Инженерная часть}{}{}{}
    \projecttask{\projectsubtasknum}
    {Описать назначение и структуру файлов папки \texttt{config}.}%
    {Текст ПЗ}%
    {11.11.2025}
    {
        \signat[img=ManilovAV, xsign=0pt, ysign=0pt, scale=0.12, xdate=10pt, ydate=-20pt, date=12.11.2025]
    }

    % ------------------------------------------------------------
    \projecttask{\projectsubtasknum}
    {Описать процесс включения в документ различных элементов.}%
    {Текст ПЗ}%
    {18.11.2025}
    {
        \signat[img=ManilovAV, xsign=0pt, ysign=0pt, scale=0.12, xdate=10pt, ydate=-20pt,  date=20.11.2025]
    }

    % ------------------------------------------------------------
    \projecttask{\projectsubtasknum}
    {Описать использование отсканированных титульных листов вместо сверстанных шаблонных титульников.}%
    {Текст ПЗ, примеры документов}%
    {25.11.2025}
    {
        \signat[img=ManilovAV, xsign=0pt, ysign=-10pt, scale=0.15, xdate=10pt, ydate=-42pt, date=27.11.2025]
    }

    % ------------------------------------------------------------
    \projecttask{\projectsubtasknum}
    {Описать процесс создания ссылок на различные элементы документа.}%
    {Текст ПЗ}%
    {02.12.2025}
    {
        \signat[img=ManilovAV, xsign=0pt, ysign=0pt, scale=0.12, xdate=10pt, ydate=-20pt, date=04.12.2025]
    }

    % ------------------------------------------------------------
    % ------------------------------------------------------------
    % --- Технологическая и практическая часть ---------------------------------
    \projecttask{\bfseries\projecttasknum}{\bfseries Технологическая и практическая часть}{}{}{}
    \projecttask{\projectsubtasknum}
    {Описать процесс сборки документов по шаблону: установку необходимого дистрибутива \LaTeX{} и подключаемых пакетов.}%
    {Текст ПЗ}%
    {09.12.2025}
    {
        \signat[img=ManilovAV, xsign=0pt, ysign=-10pt, scale=0.15, xdate=10pt, ydate=-62pt, date=11.12.2025]
    }
    % ------------------------------------------------------------
    \projecttask{\projectsubtasknum}
    {Описать использование VS Code и настройку расширения LaTeX~Workshop в IDE.}%
    {Текст ПЗ}%
    {16.12.2025}
    {
        \signat[img=ManilovAV, xsign=0pt, ysign=-10pt, scale=0.15, xdate=10pt, ydate=-42pt, date=16.12.2025]
    }
    % ------------------------------------------------------------
    \projecttask{\bfseries\projecttasknum}
    {\itshape
        Оформить пояснительную записку (ПЗ) и
        иллюстративный материал для доклада
    }%
    {Текст ПЗ, презентация}%
    {22.12.2025}
    {
        % Подпись закомментирована: задача еще не выполнена или подпись не требуется
        % \signat[img=ManilovAV, xsign=0pt, ysign=0pt, scale=0.12, xdate=10pt, ydate=-20pt, date=22.12.2025]
    }
}


% --- Список литературы -------------------------------------------------------
% Список литературы, которая должна быть включена в работу. Используется
% команда \nocite для включения источников в библиографию без явных ссылок
% в тексте задания.
\taskliterature{
    \nocite{
        kottwitzLaTeXBeginnersGuide2021,
        volnovISPOLZOVANIEIZDATELSKOYSISTEMY2022,
        mihaylovichKompyuternayaTipografiyaLaTeX2008
    }
}



\begin{document}

% --- Титульный лист ПЗ -------------------------------------------------------
% Условная вставка: либо готовый PDF, либо LaTeX-верстка
\ifusepdfPzTitle
    \includepdf[pages={-}, offset=0mm -0mm]{\pztitlepdf}
\else
    % ============================================================================
%  PZ title page: титульный лист документа "Пояснительная записка".
% ============================================================================
% Содержит шапку, тему работы, информацию об участниках с подписями и оценками,
% блок с оценками, информацию о комиссии, а также год и место выполнения работы.

\thispagestyle{empty}
\begin{center}
  {\scriptsize
    \uppercase{Министерство науки и высшего образования российской федерации}\linebreak
    \uppercase{Федеральное государственное автономное образовательное
      учреждение высшего образования}
  }

  \textbf{Национальный исследовательский ядерный университет «МИФИ»}

  {\footnotesize
    \noindent\makebox[\linewidth]{\rule{\linewidth}{0.4pt}}
  }
\end{center}

\vskip 1em

\noindent
\begin{tabular}{@{}ll@{}}
  \raisebox{-0.5\height}{\includegraphics[width=0.2\linewidth]{template/assets/mephi.png}}
   &
  \begin{tabular}{@{}c@{}}
    \textbf{\large{}Институт интеллектуальных кибернетических систем} \\
    \uppercase{\textbf{\large{}Кафедра кибернетики (№ 22)}}
  \end{tabular}
\end{tabular}


\vfill

% --- Заголовок документа ----------------------------------------------------
\begin{center}
  Направление подготовки 09.03.04 Программная инженерия

  \vfill

  {\Large{\textbf{Пояснительная записка}}}

  к \theprojecttypefulldative\space студента на тему:

  {\Large\thetitle}
\end{center}

\vfill

% --- Информация об участниках ----------------------------------------------
{\large

  \noindent
  \begin{tabularx}{\linewidth}{@{}l>{\centering}XlX@{}}
    Группа              & \centering\theauthorgroup &                & \\
    Студент             & \theauthorpzapproval      & \theauthor     & \\ \hhline{~-~}
    Руководитель        & \thesupervisorpzapproval  & \thesupervisor & \\ \hhline{~-~}
    Научный консультант & \theconsultantpzapproval  & \theconsultant & \\ \hhline{~-~}
  \end{tabularx}

  \vfill

  % --- Блок с оценками ------------------------------------------------------
  \noindent
  \begin{tabularx}{\linewidth}{@{}cX<{\centering}cX<{\centering}@{}}
    Оценка руководителя & \thesupervisorpzgrade & Оценка консультанта & \theconsultantpzgrade \\ \hhline{~-~-}
    Итоговая оценка     &                       & ECTS                &                       \\ \hhline{~-~-}
  \end{tabularx}

  \vfill

  % --- Информация о комиссии ------------------------------------------------
  % Блок для заполнения информации о комиссии (для ВКР)
  \center{Комиссия}
  \vspace{1em}

  \noindent
  \begin{tabularx}{\linewidth}{@{}lXX@{}}
    Председатель & \emptyfield & \emptyfield \\
                 & \emptyfield & \emptyfield \\
                 & \emptyfield & \emptyfield \\
                 & \emptyfield & \emptyfield \\
  \end{tabularx}

  \vfill

  % --- Место и год выполнения работы ----------------------------------------
  \begin{center}
    \textbf{Москва \the\year}
  \end{center}

}

\fi
\newpage

% --- Дополнительный лист с подписями для ВКР --------------------------------
% Раскомментируйте, если требуется включить дополнительный лист с подписями
% (например, для ВКР на кафедре 22)
% \clearpage
% \includepdf[pages={-}, offset=0mm -0mm]{3-titles/title-dep22.pdf}
% \clearpage

% --- Лист задания -----------------------------------------------------------
% Условная вставка: либо готовый PDF, либо LaTeX-верстка
\ifusepdfTaskSheet
    \includepdf[pages={-}, offset=0mm -0mm]{\tasktitlepdf}
\else
    \begin{center}
  {\scriptsize
    \uppercase{Министерство науки и высшего образования российской федерации}\linebreak
    \uppercase{Федеральное государственное автономное образовательное
      учреждение высшего образования}
  }

  \textbf{Национальный исследовательский ядерный университет «МИФИ»}

  {\footnotesize
    \noindent\makebox[\linewidth]{\rule{\linewidth}{0.4pt}}
  }
\end{center}

\vskip 1em

\noindent
\begin{tabular}{@{}ll@{}}
  \raisebox{-0.5\height}{\includegraphics[width=0.2\linewidth]{template/assets/mephi.png}}
   &
  \begin{tabular}{@{}c@{}}
    \textbf{\large{}Институт интеллектуальных кибернетических систем} \\
    \uppercase{\textbf{\large{}Кафедра кибернетики (№ 22)}}
  \end{tabular}
\end{tabular}



\begin{center}
    {\Large{\textbf{Задание на \theprojecttypeshort}}}\\

    \large

    Студенту гр. \theauthorgroup{} \theauthorfulldative
\end{center}


\begin{center}
    \uppercase{\textbf{\large{}Тема \theprojecttypeshort}}\\

    {\Large\thetitle}\\

    \vskip 1em

    \uppercase{\textbf{\large{}Задание}}
\end{center}


\begin{xltabular}{\linewidth}{|p{0.7cm}|X|>{\footnotesize\centering\arraybackslash}p{2cm}|>{\footnotesize\centering\arraybackslash}p{2cm}|>{\centering\arraybackslash}p{2.2cm}|}
    \hline
    \multicolumn{1}{|>{\centering\arraybackslash}p{0.7cm}|}{№\par п/п}
    & \multicolumn{1}{c|}{Содержание работы}
    & {\normalsize Форма \par отчетности}
    & {\normalsize Срок \par исполнения}
    & Отметка о \par выполнении \\
    \hline
    \theprojecttasks
\end{xltabular}


\refsection
\thetaskliterature
\begin{center}
    \uppercase{\textbf{\large{}Литература}}
\end{center}
\printbibliography[heading=none]
\setcounter{citenum}{0}
\endrefsection


\vfill

\begin{table}[!h]
    \captionsetup{type=table,skip=0pt}
    {\noindent\linespread{2.0}
        \begin{tabularx}{\linewidth}{p{140pt}X>{\centering}XX}
            Дата выдачи задания: & Руководитель & \thesupervisortaskapproval & \thesupervisor \\ \hhline{~~-~}
            \thetaskdate         & Студент      & \theauthortaskapproval     & \theauthor     \\ \hhline{~~-~}
        \end{tabularx}
    }
\end{table}
\fi
\newpage

% --- Настройка нумерации страниц --------------------------------------------
\pagestyle{plain}
\pagenumbering{arabic}
\setcounter{page}{2}

% --- Основное содержимое документа -------------------------------------------
% Используется отдельная refsection для изоляции библиографии ПЗ
\refsection

% --- Место для отчета Антиплагиата ------------------------------------------
% Раскомментируйте, если требуется вставить страницы с отчетом Антиплагиата
% \clearpage
% \thispagestyle{empty}
% \vfill
% \begin{center}
% [Место для распечатки отчета Антиплагиата]
% \end{center}
% \newpage
% \thispagestyle{empty}
% \vfill
% \begin{center}
% [Место для распечатки отчета Антиплагиата]
% \end{center}

\clearpage

\chapter*{Реферат}
\thispagestyle{plain}

Общий объем основного текста, без учета приложений~--- \pageref{end_of_main_text} страниц%
\ifshowannotations
    .
\else
    , с учетом приложений~--- \pageref{end_of_document} страниц.
\fi
Количество использованных источников~--- \hyperref[sec:bibliography]{\total{citenum}}.
\ifshowannotations
\else
    Количество приложений~--- \hyperref[sec:appendices]{\total{totalappendices}}.
\fi

Ключевые слова: LaTeX, шаблон, исследовательская работа, оформление, автоматизация, документация, студент.

Целью работы является \dots

В первом разделе проводится исследование проблематики, \dots

Второй раздел посвящен разработке модели \dots

В третьем разделе описывается процесс проектирования прототипа веб-сервиса, \dots

\ifshowannotations
\else
    В приложении \ref{appendix-A} пишется какой-то текст или код или картинка.
\fi


\clearpage

\input{content/1-main-content/03-toc}

\clearpage

\chapter*{Введение}
\label{sec:afterwords}
\addcontentsline{toc}{chapter}{Введение}

Впадлу писать.

\clearpage

\input{content/1-main-content/05-chapter-1}

\clearpage

\chapter{Теоретический раздел}\label{chapter2}

Тут допустим теория.

\clearpage

\chapter{Инженерно-технологический раздел}

А тут великая инженерия.

\clearpage

\chapter{Практический раздел}

А тут 100-500 скринов веб-интерфейса.

\clearpage

\input{content/1-main-content/09-chapter-5}

\clearpage

\input{content/1-main-content/10-conclusion}

% --- Подсчет общих счетчиков для реферата -----------------------------------
% Сохраняем значения счетчиков перед библиографией для использования в реферате
\setcounter{totalfigures}{\the\value{totalfigures}+\the\value{figure}}
\setcounter{figure}{0}
\setcounter{totaltables}{\the\value{totaltables}+\the\value{table}}
\setcounter{table}{0}
\setcounter{totallistings}{\the\value{totallistings}+\the\value{lstlisting}}
\setcounter{lstlisting}{0}

% Создаем метки для ссылок на общее количество фигур, таблиц и листингов
\makeatletter
\edef\@currentlabel{\the\value{totalfigures}}
\label{figures}
\edef\@currentlabel{\the\value{totaltables}}
\label{tables}
\edef\@currentlabel{\the\value{totallistings}}
\label{listings}
\makeatother

% --- Список использованной литературы ----------------------------------------
\clearpage

% ============================================================================
%  Bibliography section: вывод списка использованной литературы.
% ============================================================================

% --- Настройка вывода библиографии ------------------------------------------
% Добавляем якорь для гиперссылок и включаем список литературы в оглавление
\phantomsection
\label{sec:bibliography}
\addcontentsline{toc}{chapter}{\bibname}

% Включаем более свободное форматирование для предотвращения переполнения строк
% в библиографии (часто возникают проблемы с длинными URL и названиями)
\sloppy
\emergencystretch=3em

% Вывод библиографии
\printbibliography

% Возвращаем строгое форматирование после библиографии
\fussy
\emergencystretch=1pt

\endrefsection

\label{end_of_main_text}

% --- Приложения --------------------------------------------------------------
\clearpage
\label{sec:appendices}

\begin{appendices}
    \chapter*{Приложения}
    \addcontentsline{toc}{chapter}{Приложения}

    % --- Переопределение команд для приложений -------------------------------
    % Изменяем поведение глав, разделов и подразделов для корректного
    % попадания в оглавление и нумерации приложений буквами
    \let\oldchapter\chapter
    \renewcommand{\thechapter}{\Asbuk{chapter}}
    \renewcommand{\chapter}[1]{%
        \refstepcounter{chapter}%
        \oldchapter*{\appendixname~\thechapter.~#1}%
        \addcontentsline{toc}{section}{\appendixname~\thechapter.~#1}%
        \stepcounter{totalappendices}%
    }

    \let\oldsection\section
    \renewcommand{\section}[1]{
        \stepcounter{section}%
        \oldsection*{\thesection~#1}%
    }
    \let\oldsubsection\subsection
    \renewcommand{\subsection}[1]{
        \stepcounter{subsection}
        \oldsubsection*{\thesubsection~#1}
    }

    % --- Отключение переноса первого приложения ------------------------------
    % Первое приложение не переносится на новую страницу
    \makeatletter
    \let\oldclearpage\clearpage
    \let\clearpage\relax
    \makeatother

    % Подключение всех приложений
    % ============================================================================
%  All appendices: подключение всех приложений к документу.
% ============================================================================
% Этот файл служит точкой входа для всех приложений. Добавляйте новые
% приложения, создавая файлы вида N-appendix-Name.tex и подключая их здесь.

\chapter{Показатели XiYan-SQL в ключевых бенчмарках задачи Text-to-SQL}\label{appendix-A}

\begin{figure}[h]
	\centering
	\begin{subcaptiongroup}
		\includegraphics[width=0.85\textwidth]{literature-review/SOTA-Spider.png}
		\caption{Бенчмарк Spider}
		\label{fig:SOTA-Spider}
		\includegraphics[width=0.85\textwidth]{literature-review/SOTA-BIRD.png}
		\caption{Бенчмарк BIRD}
		\label{fig:SOTA-Bird}
	\end{subcaptiongroup}
	\captionsetup{subrefformat=parens}
	\caption{Графики лидеров на бенчмарках Spider и BIRD}
	\label{fig:SOTA-Spider-Bird}
\end{figure}

%Далее приложения начинаются с новых страниц
\makeatletter
\let\clearpage\oldclearpage
\makeatother

\clearpage

\chapter{Фрагменты программного кода}\label{appendix-B}

\lstinputlisting[
	language=Python,
	label=lst:main_app,
	frame=lines,
	breaklines=true,
	caption={Точка входа в приложение (\texttt{main.py})}
]{content/0-assets/listings/main.py}

\lstinputlisting[
	language=Python,
	label=lst:user_register,
	float=tb,frame=lines,
	caption=Реализация эндпоинта регистрации пользователя (\texttt{features/users/api.py})
]{content/0-assets/listings/user_register.py}

\lstinputlisting[
	language=Python,
	label=lst:table_upload,
	float=tb,frame=lines,
	caption=Реализация эндпоинта загрузки таблицы (\texttt{features/tables/api.py})
]{content/0-assets/listings/table_upload.py}

\lstinputlisting[
	language=Python,
	label=lst:table_preview,
	float=tb,frame=lines,
	caption=Реализация сервисной функции для получения превью таблицы (\texttt{services/table\_service.py})
]{content/0-assets/listings/table_preview.py}

\lstinputlisting[
	language=Python,
	label=lst:nli-core-mock,
	float=tb,frame=lines,
	caption=Реализация модуля-эмулятора ядра NLIDB (\texttt{services/text\_to\_sql\_service.py})
]{content/0-assets/listings/nli_core_mock.py}

\clearpage

\chapter{Вывод терминала об отчете тестирования прототипа}\label{appendix-C}

\lstinputlisting[
	language=bash,
	label=lst:test_report,
	frame=lines,
	breaklines=true,
	caption={Вывод терминала при запуске тестирования прототипа (\texttt{test-report.sh})}
]{content/0-assets/listings/test-report.sh}

    % --- Восстановление стандартного поведения ------------------------------
    % Восстанавливаем все команды как было до переопределения
    \let\chapter\oldchapter
    \let\section\oldsection
    \let\subsection\oldsubsection
\end{appendices}

\label{end_of_document}

\end{document}

